\documentclass[11pt, a4paper, german, twoside, cleardoubleempty, openright, pointlessnumbers]{scrreprt} %draft
\usepackage[ngerman]{babel}
\usepackage{setspace}
\usepackage{amsmath}
\usepackage{amssymb}
\usepackage[utf8]{inputenc} %tex file input incoding für UTF-8
\usepackage[T1]{fontenc}
\usepackage{geometry}
%\usepackage{subfigure}
%\usepackage{algorithmic}
\usepackage{placeins}
\usepackage{threeparttable}
\usepackage{courier}
%\usepackage{siunitx} wäre für Einheiten sinnvoll, aber nicht gefunden
\usepackage{epsfig}
\usepackage{booktabs}

%\usepackage{bytefield}

%\usepackage[default,osfigures,scale=0.95]{opensans} %TU Schriftart

\usepackage{tikz}
\usetikzlibrary{positioning}
\usetikzlibrary{fit}
\usetikzlibrary{calc,intersections}
\usetikzlibrary{circuits.logic}


\usepackage[automark,headsepline]{scrpage2}
\pagestyle{scrheadings}

\clearscrheadfoot
\automark[section]{chapter}
\lehead[]{\leftmark}
\rohead[]{\rightmark}
\ofoot[\pagemark]{\pagemark}

\usepackage{listings}
\lstset{
	belowcaptionskip=1\baselineskip,
	breaklines=true,
	frame=L,
	xleftmargin=\parindent,
	language=C,
	numbers=left,
	showstringspaces=false,
	basicstyle=\footnotesize\ttfamily,
	keywordstyle=\bfseries\color{green!40!black},
	commentstyle=\itshape\color{purple!40!black},
	identifierstyle=\color{blue},
	stringstyle=\color{orange},
	escapeinside={(@}{@)}
}
\usepackage{color}
\usepackage{graphicx}
\usepackage{float}
\usepackage[right]{eurosym}%Fuer Euro-Symbol
\geometry{left=25mm,right=25mm,top=30mm,bottom=3cm}
\setlength{\parindent}{0pt}
\parskip 6pt
\usepackage{lscape}
\usepackage[printonlyused]{acronym}
\usepackage{caption}
%\usepackage{ccaption}
%\captionstyle{\centering}
\usepackage[labelfont=bf,textfont=normalfont,singlelinecheck=off,justification=centering]{subcaption}
\usepackage{hyperref}

\hypersetup{
    bookmarks=true,         			% show bookmarks bar?
    unicode=false,          			% non-Latin characters in Acrobat?s bookmarks
    pdftoolbar=true,        			% show Acrobat?s toolbar?
    pdfmenubar=true,        			% show Acrobat?s menu?
    pdffitwindow=false,     			% window fit to page when opened
    pdfstartview={FitH},    			% fits the width of the page to the window
    pdftitle={Titel},                   % title
    pdfauthor={},     	                % author
    pdfsubject={},   				    % subject of the document
    pdfcreator={},   		            % creator of the document
    pdfproducer={}, 	                % producer of the document
    pdfkeywords={}, 				    % list of keywords
    pdfnewwindow=true,     			    % links in new window
    colorlinks=true,       			    % false: boxed links; true: colored links
    allcolors=black,
    linkcolor=black,          			% color of internal links
    %color=black, broken        		% color of links to bibliography
    filecolor=black,      				% color of file links
    urlcolor=black,           			% color of external links
    pdfstartpage=1
}

%------------------Anfang Nummerierung Anhang-----------------
 \renewcommand\appendix{\par 
   \setcounter{section}{0}% 
   \setcounter{subsection}{0}% 
   \setcounter{figure}{0}%s
   \renewcommandsection{\Alph{section}}% 
   \renewcommand\thefigure{\Alph{section}.\arabic{figure}}} 
%------------------Ende Nummerierung Anhang-----------------

\begin{document}
	\bibliographystyle{alphadin}
		
	\begin{center}
	\thispagestyle{empty}
	\mbox{}
	\vspace{3\baselineskip}
	
	\textsc{\Large Technische Universität Dresden\\
		Fakultät Informatik\\
		Institut für Technische Informatik\\
		Professur für VLSI-Entwurfssysteme, Diagnostik und Architektur\\ \vspace{2\baselineskip} }
		
	\vspace{2\baselineskip}
		
	\textbf{\LARGE Bachelorarbeit \vspace{\baselineskip}}
	
	\vspace{2\baselineskip}
	
	\Large Evaluation einer modernen Zynq-Plattform am Beispiel der
	Implementierung einer Hough Transformation

	\vspace{2\baselineskip}
		
	\large Dominik Weinrich\\
	\small geboren am 11.08.1990 in Kassel\\
	\small (Mat.-Nr.: 3914410)\\
	\vspace{2\baselineskip}


	\end{center}

	\vspace{6\baselineskip}
	
	\begin{minipage}[t]{8cm}
		{\small Betreuender Hochschullehrer:}\\\large Prof. Dr.-Ing. habil. Rainer G. Spallek\\
	\end{minipage}
	
	\begin{minipage}[t]{8cm}
	{\small Betreuer:}\\\large Oliver Knodel\\
	\end{minipage}

	\vspace{2\baselineskip}


	Dresden, Datum
	
	
	\newpage
	
	\setstretch{1,5}
	
	%---Leerseite--------->
	\newpage
	\mbox{}
	\thispagestyle{empty}
	\newpage
	%--------------------->
	
	
\setcounter{page}{1}
\pagenumbering{Roman}

\chapter*{Aufgabenstellung}
%\includegraphics[width=\linewidth]{Bilder/Aufgabenstellung_s.png}
%UT Aufgabenstellung einfügen

	%---Leerseite--------->
	\newpage
	\mbox{}
	\thispagestyle{empty}
	\newpage
	%--------------------->

\setstretch{1,2}	
\begin{minipage}[t]{16cm}
\chapter*{Selbstständigkeitserklärung}	
Ich versichere, dass ich die vorliegende Studienarbeit zum Thema 

\begin{center}
	\bf{Evaluation einer modernen Zynq-Plattform am Beispiel der
		Implementierung einer Hough Transformation} 
\end{center}

selbstständig verfasst und
ohne Benutzung anderer 
als der angegebenen Hilfsmittel angefertigt, nur die angegebenen Quellen benutzt
und die in den benutzten Quellen wörtlich oder inhaltlich entnommenen Stellen
als solche kenntlich gemacht habe. Die Arbeit wurde in gleicher oder ähnlicher
Form noch keiner anderen Prüfungsbehörde vorgelegt.

\vspace{3\baselineskip}

Name, Dresden, Datum
\end{minipage}

\vspace{5\baselineskip}

\begin{minipage}[t]{16cm}
\chapter*{Wettbewerbsrechtlicher Hinweis}
Die bloße Nennung von Namen, Produkten, Herstellern und Firmennamen dient 
lediglich als Information und stellt keine Verwendung des Warenzeichens sowie
keine Empfehlung des Produktes oder der Firma dar. 
\end{minipage}

	\newpage
	\markboth{}{}
	\setcounter{tocdepth}{3} %gliederung inhaltsverzeichnis
	\setcounter{secnumdepth}{3} %nummerierung tiefe
	\setstretch{1,3}% damit es genau auf die Seite passt
	 \tableofcontents

	\markboth{}{}

	%---Leerseite--------->
	\newpage
	\mbox{}
	\thispagestyle{empty}
	\newpage
	%--------------------->
	
	\setstretch{1,2}
	\listoffigures
	\addcontentsline{toc}{chapter}{Abbildungsverzeichnis}

	\newpage
	
	%\begin{minipage}[t]{16cm}
		\listoftables
	%\end{minipage}
	\addcontentsline{toc}{chapter}{Tabellenverzeichnis}
	
	\newpage
	
	%\begin{minipage}[t]{16cm}
		\lstlistoflistings
	%\end{minipage}
	\addcontentsline{toc}{chapter}{Listings}
	
	%-------------Abkuerzungen
	\setstretch{1,2}
	\chapter*{Abkürzungsverzeichnis}
	\addcontentsline{toc}{chapter}{Abkürzungsverzeichnis}
	\begin{acronym}[]
	%Einleitung
	%a
	\acro{ABS}{Antiblockiersystem}
	\acro{APU}{Application Processing Unit}
	
	%b
	\acro{BRAM}{Block RAM}
	
	%c
	\acro{CPU}{Central Processing Unit}
	\acro{CHT}{Circle Hough Transformation}
	
	%d
	\acro{DSP}{Digital Signaling Processor}

	%e
	\acro{ESP}{Elektronisches Stabilitätsprogramm}
	
	%f
	\acro{FPGA}{Field Programmable Gate Array}

	%g
	\acro{GPU}{Graphics Processing Unit}
	
	%h
	\acro{HLS}{High Level Synthese}
	\acro{HT}{Hough-Transformation}
	
	%i

	%j
	
	%k
	
	%l
	\acro{LUT}{Look Up Table}
	
	%m
	\acro{MPSoC}{Multi Processor System on a Chip}

	%n
	
	%o
	\acro{OMP}{Open MP}
	
	%p
	\acro{PL}{User-Programmable Logic}
	\acro{PS}{Processing System}

	%q
	
	%r
	\acro{RAM}{Random Access Memory}
	\acro{RPU}{Realtime Processing Unit}
	\acro{RTL}{Register Transfer Level}

	%s
	\acro{SDL 2}{Simple DirectMedia Layer 2}
	\acro{SoC}{System on a Chip}

	%t
	
	%u

	%v

	%w
	
	%x
	
	%y
	
	%z
\end{acronym}

	
	\clearscrheadfoot
	\automark[section]{chapter}
	\lehead[]{\leftmark}
	\rohead[]{\rightmark}
	\ofoot[\pagemark]{\pagemark}
	
	%---Leerseite--------->
	\newpage
	\mbox{}
	\thispagestyle{empty}
	\newpage
	%--------------------->
	
	
	%---------------------------------------------------------------------------------------------------------------------------------------------------------------------
	\pagenumbering{arabic}
	\setcounter{page}{1}%Am Ende noch mal Kontrollieren!!!!
	\setstretch{1,2}
	
	\chapter{Einleitung und Motivation}
	Vor einigen Jahren hatten die meisten Autos ein \ac{ABS} und ein \ac{ESP} zur Unterstützung bei Gefahrenbremsungen und zur Stabilisierung der Fahrt in scharfen Kurven. Mittlerweile ist die Liste der Fahrassistenzsysteme deutlich größer geworden und umfasst unter anderen auch Adaptive Geschwindigkeitsregelanlagen, Spurhalte- und Spurwechselassistenten, Einparkhilfen und Autonome Notbremssysteme. Viele dieser Systeme basieren auf Bildbearbeitungs- und Bildanalysealgorithmen, welche gerade in Gefahrensituationen schnellstmöglich ausgewertet werden müssen. Hierfür ist die Nutzung einer \ac{CPU} meist keine ausreichende Lösung, da diese zwar eine vergleichsweise hohe Taktrate besitzt, aber aktuell mit bis zu 8 Kernen nicht genug Parallelität bietet, um große Bilder effektiv auswerten zu können. Solche Probleme werden daher vermehrt in Hardware ausgelagert durch einen \ac{FPGA} oder eine \ac{GPU} gelöst.
\\
Diese Bachelorarbeit behandelt eine solche Auslagerung von Software in Hardware. Dazu wird im zweiten Kapitel einleitend die Zielarchitektur vorgestellt und ein Einblick in die Grundlagen des Hardware/Software Codesigns, so wie der \ac{HLS} gegeben. Im dritten Kapitel wird am Beispiel einer Hough Transformation eine \ac{HLS} durchgeführt. Dazu wird zunächst die Softwareimplementierung vorgestellt. Anschließend werden iterativ einzelne Komponenten der Hough Transformation von einer \ac{CPU} auf einen \ac{FPGA} ausgelagert.
\\
Das vierte Kapitel behandelt die Auswertung des Hardware/Software Codesigns hinsichtlich des erbrachten Speedups und des Ressourcenverbrauchs und im fünften Kapitel wird abschließend ein Fazit gezogen und ein Ausblick in das Thema gegeben.

	\chapter{Grundlagen}
	In dem folgenden Kapitel wird näher auf die Grundlagen eingegangen, welche für diese Arbeit benötigt werden. Zunächst wird die verwendete Hardware vorgestellt und ein Einblick in die Grundlagen des \ac{HW}/\ac{SW}-Codesigns gegeben. Anschließend wird kurz auf die \ac{HLS} eingegangen und zuletzt werden die benötigten Algorithmen zur Bildbearbeitung betrachtet.

	\section{Zielarchitektur - Zynq Systemarchitektur}
	Als Zielarchitektur für diese Arbeit wird das Xilinx Zynq UltraScale+ MPSoC ZCU102 Evaluation Board verwendet. Es besitzt ein \ac{MPSoC} mit einer \ac{PL} und einem \ac{PS}, bestehend aus mehreren Prozessoren.
\\
\\
Der rekonfigurierbare Teil besteht aus einem Zynq UltraScale XCZU9EG-2FFVB1156 FPGA mit 600 logischen Einheiten, 32.1 Mb Speicher, 2.520 \acp{DSP} und 328 I/O Pins. Dem FPGA stehen 512 MB DDR4 \ac{RAM} bei 1200 MHz / 2400 Mbps zur Verfügung.
\\
\\
Das \ac{PS} beinhaltet eine \ac{APU}, zwei \acp{RPU} und eine \ac{GPU}. Es befinden sich ein Cortex-A53 64-bit Vierkernprozessor mit jeweils 32 KB L1 Cache und 1 MB L2 Cache auf dem \ac{MPSoC}, sowie zwei Cortex-R5 \acp{RPU}. Außerdem verfügt das Board über eine Mali-400 MP2 \ac{GPU} mit 64 KB L2 Cache. Den Prozessoren stehen 4 GB DDR4 \ac{RAM} zur Verfügung.
\cite{zcu102ug}

\begin{figure}[htb]
	\centering
	\includegraphics[width=\linewidth]{Bilder/zynq-zcu102ug.pdf}
	\caption[Zynq Ultrascale+ MPSoC Block Diagramm]{ZynQ Ultrascale+ MPSoC Bock Diagramm\\Quelle: \cite[S. 21]{zcu102ug}}
\end{figure}

	\section{Hardware/Software Codesign}
	Hardware/Software Codesign bezeichnet Entwurfsmethoden für elektronische Systeme, welche die jeweiligen Vorteile von Hardware und Software ausnutzen. Typischerweise ist die Anwendung in Softwarekomponenten, die auf den Prozessorkernen laufen, und Hardwarekomponenten, welche die Anwendung beschleunigen oder Schnittstellen für die Umgebung bereitstellen, eingeteilt.
\cite[S.5]{ha2017hwswco}

	\section{High Level Synthese}
	In den vergangenen Jahrzehnten ist die Komplexität von Hardware und integrierten Schaltungen immer weiter angestiegen. Um eine einfachere Entwicklung dessen zu ermöglichen, wurde es notwendig, neue Entwurfsmethoden auf einem höheren Abstraktionslevel einzuführen. Die \ac{HLS} ist ein automatisierter Entwurfsprozess, der aus Programmcode digitale Hardware generiert. Diese erfüllt die Funktionalität des Codes.
\\
In der Softwarebranche wurden ähnliche Schritte mit der Ablösung von Maschinencode durch Assembler und später durch Hochsprachen wie C durchgeführt. Auf Grund der Komplexität heutiger Software ist es nahezu undenkbar, diese vollständig in Assembler zu entwickeln.
\cite[S. 8]{coussy2009hls}
\\
\\
Der erste Schritt einer \ac{HLS} besteht aus der Kompilierung des Codes zu Kontroll- und Datenflussgraphen. Dabei werden zumeist auch verschiedene Optimierungen vorgenommen. Darunter fallen beispielsweise die Eliminierung von unerreichbarem Code, Schleifenoptimierungen und die Substitution der Ergebnisse von Berechnungen durch Konstanten.
\\
Anschließend werden Typ und Anzahl der benötigten Hardware Ressourcen bestimmt. Diese sind z.B. Speichereinheiten, Busse oder funktionale Einheiten. Sie werden aus einer \ac{RTL} Bibliothek ausgewählt. Darin sind verschiedene Komponenten sowie deren Spezifikationen (Größe, Delay, benötigte Energie) enthalten.
\\
Im nächsten Schritt findet das \emph{Scheduling} von verschiedenen Operationen des Typs $a = b \odot c$ statt. Dabei steht $\odot$ für einen beliebigen Operator. Die Variablen $b$ und $c$ müssen entweder aus einem Speicher oder aus der Ausgabe einer funktionalen Einheit geladen werden. Operationen können verkettet oder parallel ausgeführt werden, falls es keine Datenabhängigkeiten gibt.
\\
Die benötigten Variablen werden dann an Speichereinheiten gebunden. Sofern sich die \emph{Lebenszeiten} von verschiedenen Variablen nicht überlappen, können diese an die selbe Speichereinheit gebunden werden. Außerdem ist es notwendig, dass Operationen an funktionale Einheiten gebunden werden, welche in der Lage sind, die Operation auszuführen. Wenn es mehrere solcher Einheiten gibt, optimiert der \emph{Binding Algorithmus} die Auswahl. Die Einheiten werden bei dem beschriebenen Verfahren durch Busse miteinander verbunden.
\\
Im letzten Schritt der \ac{HLS} wird schließlich ein \ac{RTL} Modell des synthetisierten Designs generiert.
\cite[S. 9-11]{coussy2009hls}

	\section{Pipelining}
	%Das Konzept des Pipelinings kann auf Funktionen und Schleifen angewandt werden, um diese zu Beschleunigen. Dies ermöglicht es Operationen parallel ausgeführt zu werden.
\\
\\
In dem Beispiel aus \autoref{img:pipelining} ist die Schleife in drei Operationen unterteilt. Die erste Operation op\_Read liest Daten aus dem Speicher, die zweite Operation op\_Compute führt eine Berechnung aus und die dritte Operation op\_Write schreibt Daten in den Speicher.
\\
Der linke Teil der Grafik zeigt die Laufzeit für die serielle Verarbeitung der Schleife. Es dauert 3 Takte, bis neue Daten gelesen werden können und nach 2 Takten ist das Ergebnis der Operation von einer Iteration berechnet. Nach 8 Takten sind die Ergebnisse aller Operationen berechnet.
\\
Wenn die Schleife allerdings gepipelined wird, kann wie im rechten Teil der Grafik zu erkennen, in jedem Takt eine neue Leseoperation angefangen werden. Damit stehen bereits nach 4 Takten die Ergebnisse aller Operationen zur Verfügung.
\\
Das Pipelining funktioniert nur, sofern keine Datenabhängigkeiten vorhanden sind. Das bedeutet, dass in der Berechnung keine Daten benötigt werden, die in dem vorherigen Schleifendurchlauf berechnet werden. Diese sind noch nicht zurück in den Speicher geschrieben und damit noch nicht vorhanden.

\begin{figure}[H]
	\centering
	\includegraphics[width=0.8\linewidth]{Bilder/pipelining.pdf}
	\caption[Pipelining]{Pipelining einer Schleife \cite[S. 190]{vivado902ug}}
	\label{img:pipelining}
\end{figure}

	\section{Bresenham Algorithmus}
	Mit dem Bresenham Algorithmus können die Punkte eines Kreises mit gegebenem Radius $r$ und dem Mittelpunkt $(x_0, y_0)$ berechnet werden. Da der Algorithmus nur auf ganzzahligen Berechnungen basiert, ist er besonders effektiv und zählt zu den Standardalgorithmen in der Bildbearbeitung.
\\
\\
Auf Grund der Symmetrie eines Kreises kann dieser in acht Oktanten unterteilt werden. Der Bresenham Algorithmus zeichnet bei jedem Schritt für jeden Oktanten ein Pixel.
\\
Am Anfang können außerdem bereits die Punkte $(x_0+r, y_0)$, $(x_0-r, y_0)$, $(x_0, y_0+r)$ und $(x_0, y_0-r)$ eingezeichnet werden. Im Folgenden wird der Algorithmus für den ersten Oktanten gezeigt. Bei den anderen Oktanten kann analog verfahren werden.
\\
\\
Für den ersten Oktanten zeichnet der Algorithmus in jedem Schritt ein Pixel an der Position $(x_0+x, y_0+y)$. Dabei sind x und y Laufvariablen, die sich mit jedem Schritt verändern. Es gibt eine schnelle und eine langsame Laufrichtung. Die schnelle Richtung ist im Falle des ersten Oktanten die y-Richtung. Diese wird mit 0 initialisiert und bei jedem Schritt um eins erhöht. Die langsame Richtung, also die x-Richtung, wird mit $r$ initialisiert und abhängig von einer Fehlervariablen dekrementiert. Der Fehler wird zu Beginn auf den Radius gesetzt. Bei jedem Schritt wird $y*2 + 1$ von dem Fehler abgezogen. Wenn die Fehlervariable kleiner als 0 ist wird die langsame Richtung um eins verringert und $1 - 2 \cdot x$ vom Fehler abgezogen. Der Algorithmus terminiert, wenn die schnelle Richtung größer oder gleich der langsamen Richtung ist, also wenn $y \ge x$ gilt. Die Anzahl der erzeugten Punkte für einen Kreis beträgt: $4 \cdot \sqrt{2} \cdot r$.
\cite{bresenham1977circle}

%TODO: neues Bild, eigenes Bild?
\begin{figure}[H]
	\centering
	\includegraphics[width=0.5\linewidth]{Bilder/Bresenham_circle.png}
	\caption[Bresenham Algorithmus]{Bresenham Algorithmus $[$https://upload.wikimedia.org/wikipedia/de/0/09/Bresenham\_circle2.png, 02.07.2018$]$}
	\label{img:bresenham}
\end{figure}

	\section{Hough Transformation}
	Die Hough-Transformation ist ein Verfahren zur Erkennung von beliebigen parametrisierbaren Objekten in einem Bild. Dabei können durch ein Votingverfahren auch unvollständige, z.B. teilweise verdeckte Objekte, erkannt werden.
\\
Da die Hough-Transformation nur auf binäre Gradientenbilder angewandt werden kann, muss ein Eingabebild zunächst in ein solches umgewandelt werden. Hierfür sind mehrere Schritte erforderlich, die in den folgenden Unterkapiteln behandelt werden.
	\subsection{Umwandlung eines RGB Bildes in Graustufen}\label{sec:umwandlung-eines-rgb-bildes-in-graustufen}
	Damit eine Kantenextraktion mit dem Canny-Algorithmus (\ref{canny}) durchgeführt und somit ein binäres Gradientenbild erzeugt werden kann, muss zuvor eine Umwandlung eines Farbbildes in ein Graustufenbild erfolgen. Ein Bild ist grau, wenn die rot, grün und blau $(R, G, B)$ Komponenten den selben Wert haben. Dann kann allerdings auch anstatt der zuvor benötigten drei Farbkanäle $(R, G, B)$ mit je 8 Bit pro Pixel nur noch ein Farbkanal mit 8 Bit pro Pixel verwendet werden. Damit wird die Größe des Bildes und damit auch die Größe, der zu verarbeitenden Daten deutlich reduziert.
\\
\\
Für jedes Pixel werden die drei Farbkomponenten $(R, G, B)$ in einen Intensitätswert $Y$ umgerechnet, welcher die Helligkeit des Pixels beschreibt. Es gibt eine Vielzahl verschiedener Ansätze, die sich unterschiedlich gut zur Kantenextraktion eignen. An dieser Stelle soll nicht näher auf die Unterschiede eingegangen werden. Zwei detaillierte Vergleiche zu dem Thema finden sich in \cite{kanan2012grayscale} und \cite{ahmad2018grayscale}. Daraus geht hervor, dass GLuminance als Standardalgorithmus zur Bildverarbeitung gilt. Die Intensität berechnet sich nach GLuminance gewichtet, da wir rot und grün deutlich heller als blau wahrnehmen.
\\
\begin{equation}
Y = 0.3 * R + 0.59 * G + 0.11 * B
\end{equation}
\\
Die Umwandlung eines Bildes der Größe $M \times N$ benötigt $5(M \cdot N)$ Operationen. Unter der vereinfachten Annahme, dass die Bildgröße $N \times N$ ist folgt damit eine Komplexität von $O(N^2)$.

	\subsection{Gauß-Filter}\label{sec:gaus-filter}
	Einige Bilder enthalten ein Bildrauschen, welches bei einer späteren Kantenextraktion zu fehlerhaften Kanten führen kann. Um dieses Rauschen zu verringern wird zur Glättung des Bildes ein Gauß-Filter angewandt. Dieses addiert, abhängig von seiner Größe, benachbarte Pixel gewichtet auf. Das zu filternde Pixel hat hierbei die höchste Gewichtung. Umso weiter ein Pixel von diesem entfernt ist, desto kleiner wird dessen Gewichtung. Anschließend wird der Wert normiert, indem er durch die Summe aller Gewichtungen geteilt wird.
\begin{align}
h(x) = \frac{1}{\sigma \sqrt{2\pi}} e^{-\frac{x^2}{2 \sigma^2}}
&&
h(x,y) = \frac{1}{2 \pi \sigma^2} e^{-\frac{x^2+y^2}{2 \sigma^2}}
\end{align}
Die eindimensionale Impulsantwort $h(x)$ entspricht der Funktion der Normalverteilung. Die zweidimensionale Impulsantwort $h(x,y)$ des Gauß-Filters ergibt sich aus dem Produkt der Impulsantworten in x- und y-Richtung. Die Argumente x und y bezeichnen jeweils die Entfernungen in horizontaler und vertikaler Richtung zum Ursprung und $\sigma$ bezeichnet die Standardabweichung der Normalverteilung.
\\
Ein Filterkernel für einen Gauß-Filter lässt sich nun über die zweidimensionale Impulsantwort berechnen. Ein Kernel der Größe $N \times N$ mit $N\in\{3, 5, ...\}$ berechnet sich wie folgt:
\begin{align*}
&A = \sum_{x=-B}^{B} \sum_{y=-B}^{B} h(x,y) \\
&B = (N-1)/2
\end{align*}
\begin{equation}
\frac{1}{A}
\begin{pmatrix}
h(-B,-B) & ... & ... & h(0,-B) & ... & ... & h(B,-B) \\
h(-B,-1) & ... & h(-1,-1) & h(0,-1) & h(1, -1) & ... & h(B,-1) \\
h(-B,0) & ... & h(-1,0) & h(0,0) & h(1,0) & ... & h(B,0) \\
h(-B,1) & ... & h(-1,1) & h(0,1) & h(1,1) & ... & h(B,1) \\
h(-B,B) & ... & ... & h(0,B) & ... & ... & h(B,B)
\end{pmatrix}
\end{equation}
In der Bildverarbeitung wird häufig nur eine Approximation eines Gaußkernels verwendet, die auf dem Pascalschen Dreieck beruht. Diese bietet den Vorteil, dass der Kernel aus ganzen Zahlen aufgebaut ist und zur Normierung durch eine Zweierpotenz geteilt werden kann. Für einen Kernel der Größe $N$ wird die $N$-te Zeile des Pascalschen Dreiecks in ein Feld mit $N$ Elementen geladen. Anschließend wird der Kernel wie folgt aufgebaut: Die $i$-te Zeile des Kernels entsteht durch Multiplikation des Feldes mit dem $i$-ten Element des Feldes. Der Faktor zur Normierung ist dann $A=2^N$.
\\
Durch Ausnutzen der Separierbarkeit des Gauß-Filters kann die Rechenzeit weiter reduziert werden.....
	\subsection{Canny Edge Detection}\label{sec:canny-edge-detection}
	Die Canny Edge Detection Methode ist eine der am meisten verbreiteten Kantenerkennungsalgorithmen. Er erzielt eine gute Lokalisierung von Kanten, minimiert die Erkennung von falschen Kanten und beschränkt die Kantenbreite auf ein Pixel.
\cite[S.132]{burger_2016dip_java}
\\
\\
Kanten sind Stellen in Bildern, an denen sich die Intensität zu einer Richtung hin stark verändert. Je größer die Veränderung ist, desto wahrscheinlicher befindet sich eine Kante an der Position. Um die Kanten in Bildern zu finden muss nach großen Differenzen zwischen den Intensitäten benachbarter Pixel gesucht werden. Dies entspricht großen Werten in der ersten Ableitung, welche über ein Gradientenfilter berechnet werden kann. Der Sobeloperator (\ref{sobel}) ist das gängigste Filter.
\cite[S.122-125]{burger_2016dip_java}
\\
\begin{align}\label{sobel}
S_x =
\begin{bmatrix}
-1 & 0 & 1 \\
-2 & 0 & 2 \\
-1 & 0 & 1
\end{bmatrix}
\qquad
S_y =
\begin{bmatrix}
-1 & -2 & -1 \\
 0 &  0 &  0 \\
 1 &  2 &  1
\end{bmatrix}
\end{align}
\\
Der Sobeloperator wird auf jedes Pixel in x- und y-Richtung angewandt. Anschließend berechnet sich die Kantenstärke über den euklidschen Betrag der partiellen Ableitungen $g_x(x,y)$ und $g_y(x,y)$.
\cite[S.134]{burger_2016dip_java}
\\
\begin{equation}
g(x,y) = \sqrt{g_x(x,y)^2 + g_y(x,y)^2}
\end{equation}
\\
Als Approximation werden auch häufig nur die Beträge der partiellen Ableitungen addiert.
\cite{QUELLE FEHLT}
\\
\begin{equation}
g(x,y) = |g_x(x,y)| + |g_y(x,y)|
\end{equation}
\\
Edge Localization (Non Maximum Suppression)

	\subsection{Circle Hough Transformation}\label{sec:circle-hough-transformation}
	Die \ac{CHT} ist eine Variation der \ac{HT} zur Erkennung von Kreisen in Bildern. Dabei werden als Parameter zur eindeutigen Beschreibung des Kreises die Koordinaten des Mittelpunktes $(x_0,y_0)$ und der Radius $r$ verwendet. Als Eingabe benötigt der Algorithmus ein binäres Gradientenbild, welches z.B. durch eine vorherige Kantenerkennung erzeugt wurde.
\\
\\
Ein Punkt $p = (x,y)$ liegt auf einem Kreis mit den gegebenen Parametern, wenn die folgende Gleichung erfüllt ist.
\\
\begin{equation}
(x-x_0)^2+(y-y_0)^2 = r^2
\end{equation}
\\
Daher benötigt die \ac{CHT} einen dreidimensionalen Parameterraum $A(x_0,y_0,r)$, um Position und Radius von Kreisen in Bildern zu finden. Es existiert keine einfache funktionale Abhängigkeit zwischen den Koordinaten im Parameterraum.
Voting erklären usw....
\cite[S. 64]{burger2009podip_2}

	%\subsection{Optimierungen}
	%\input{Inhalt/Grundlagen/Optimierungen}
	\chapter{Hardware/Software Codesign am Beispiel einer Hough Transformation}
	Bei der gegebenen Zielarchitektur besteht das Hardware/Software Codesign aus dem Zusammenspiel des Cortex-A53 Prozessors und des Zynq UltraScale XCZU9EG-2FFVB1156 FPGAs. Die Hauptanwendung läuft auf dem Prozessor, während einzelne Komponenten zur Beschleunigung auf den FPGA ausgelagert werden. Das Hardware/Software Codesign wird mit dem Programm Vivado HLS, aus der Vivado Design Suite 2017.4 \cite{vivado902ug}, durchgeführt. Dazu wird im folgenden zunächst die Softwareimplementierung und anschließend die Auslagerung mittels \ac{HLS} vorgestellt.

	\section{Softwareimplementierung}
	In diesem Kapitel wird die Softwareimplementierung der \ac{HT} vorgestellt. Da das zur anschließenden \ac{HLS} verwendete Programm \emph{Vivado HLS} für C Code ausgelegt ist wurde C als Programmiersprache gewählt. Die Implementierung kann in 5 verschiedene Module unterteilt werden, welche die unterschiedlichen Funktionalitäten implementieren. Dabei ist ein modularer Aufbau zur Übersicht, Zeiterfassung und vor allem für die spätere \ac{HLS} von großem Vorteil. Es lässt sich jedes Modul einzeln synthetisieren und so in \ac{HW} auslagern. Die einzelnen Module werden in den folgenden Unterkapiteln näher beschrieben.
\\
\\
Als Compiler wurde GCC (TODO VERSIONNR.) verwendet.
\\
Zum Laden des zu verarbeitenden Bildes und der Pixeldaten wurde die Bibliothek \ac{SDL 2}\footnote{\url{https://www.libsdl.org/}} (TODO VERSIONNR.) verwendet, welche einen leichten Zugriff auf die Pixeldaten ermöglicht und es einfach gestaltet die von den Zwischenschritten erzeugten Bilder abzuspeichern.
\\
Zur Parallelisierung des Programmes wurde \ac{OMP}\footnote{\url{https://www.openmp.org/}} (TODO VERSIONNR.) genutzt. \ac{OMP} ermöglicht eine leichte Parallelisierung durch Präprozessoranweisungen und bietet über das Compilerflag \emph{-fopenmp}, welches das Präprozessormakro \emph{\_OPENMP} definiert, eine einfache Möglichkeit, um eine serielle und parallele Version des Programms zu kompilieren.

\begin{figure}[htb]
	\centering
	\includegraphics[width=\linewidth]{Bilder/usage.png}
	\caption{Benutzeranleitung für das Programm}
	\label{img:usage}
\end{figure}

Das Programm wird in Unix mit \emph{./hough} (oder \emph{./hough\_omp} für die parallele Version) ausgeführt. Die möglichen Argumente sind \autoref{img:usage} zu entnehmen.

	\subsection{Main}
	Das Main Modul ist die Steuereinheit des Programmes. Hier wird \ac{SDL 2} initialisiert und es werden die verschiedenen Argumente, die dem Programm übergeben werden können, ausgewertet. Nicht gesetzte Parameter werden mit Standardwerten initialisiert.
\\
Anschließend wird das zu verarbeitende Bild geladen und es werden nacheinander die einzelnen Module aufgerufen. Das Main Modul sorgt dafür, dass die von den anderen Modulen erzeugten Pixeldaten zwischengespeichert und an das jeweils nachfolgenden Module übergeben wird. Zusätzlich wird für jedes Modul die benötigte Laufzeit gemessen und nach erfolgreicher Ausführung ausgegeben. Es wird außerdem zur Veranschaulichung das von jedem Modul erzeugte Bild abgespeichert. Um ein visuelles Ergebnis zu erhalten, werden die von der \ac{CHT} gefundenen Kreise mit Hilfe des Bresenham Algorithmus in das von der Kantenerkennung erzeugte binäre Gradientenbild eingezeichnet und das entstandene Bild ebenfalls abgespeichert.

	\subsection{Grayscaler}
	Der Grayscaler wandelt ein RGB-Bild nach \autoref{sec:umwandlung-eines-rgb-bildes-in-graustufen} in ein Graustufenbild um. Dabei erhält er als Eingabe einen Zeiger auf 32 Bit breite Pixeldaten und die Höhe und Breite des Bildes. Anschließend wird jedes Pixel durchlaufen und in \autoref{lst:grayscaler} Z. \ref{lst:grayscaler_output} der neue Intensitätswert berechnet. Die einzelnen Werte für Rot, Grün und Blau berechnen sich in Z. \ref{lst:grayscaler_rgb_start} - \ref{lst:grayscaler_rgb_end}. Als Pixelformat wurde \emph{ARGB8888} gewählt, welches in \autoref{img:argb-format} näher dargestellt ist. Die 8 höchstwertigen Bit enthalten den Alphawert des jeweiligen Pixels, die nächsten 8 Bit den Rotwert, die folgenden 8 Bit den Grünwert und die 8 niedrigsten Bit den Blauwert.

\begin{figure}[H]
	\centering
	\includegraphics[width=0.8\linewidth]{Bilder/ARGB_format.png}
	\caption[Pixelformat ARGB]{Darstellung des Pixelformates ARGB8888}
	\label{img:argb-format}
\end{figure}

Die For-Schleifen, in welchen das gesamte Bild durchlaufen und verarbeitet wird, werden in den Z. \ref{lst:grayscaler_omp_start} - \ref{lst:grayscaler_omp_end} parallelisiert, falls \ac{OMP} verwendet wird.

\begin{lstlisting}[label=lst:grayscaler,caption=Auszug aus grayscaler.c]
uint8_t* grayscaler(uint32_t* input, unsigned int width, unsigned int height) {
uint8_t *output = malloc(width * height * sizeof(uint8_t));

#ifdef _OPENMP(@\label{lst:grayscaler_omp_start}@)
#pragma omp parallel for
#endif(@\label{lst:grayscaler_omp_end}@)
// Convert image to grayscale
for(int y = 0; y < height; y++) {(@\label{lst:grayscaler_outer_loop}@)
	for(int x = 0; x < width; x++) {(@\label{lst:grayscaler_inner_loop}@)
		uint8_t r, g, b;
		int index = y * width + x;
		
		r = input[index] >> 16 & 0xFF;(@\label{lst:grayscaler_rgb_start}@)
		g = input[index] >> 8 & 0xFF;
		b = input[index] & 0xFF;(@\label{lst:grayscaler_rgb_end}@)
		output[index] = 0.3*r + 0.59*g + 0.11*b;(@\label{lst:grayscaler_output}@)
	}
}

return output;
}
\end{lstlisting}

Als Ausgabe liefert der Grayscaler einen Zeiger auf die neu berechneten 8 Bit breiten Pixeldaten. Es muss von der Main Methode dafür gesorgt werden, dass die erzeugten Pixeldaten auch in ein geeignetes Bild geladen werden.

\begin{figure}[H]
	\centering
	\begin{subfigure}{.5\textwidth}
		\centering
		\includegraphics[width=.8\linewidth]{Bilder/euro.png}
		\caption[Eingabebild des Grayscalers]{Eingabebild des Grayscalers\\ $[$http://www.historia-hamburg.de/media/product/1ec/19-x-1-euro-satz-aus-19-euro-staaten-511.jpg$]$}
	\end{subfigure}%
	\begin{subfigure}{.5\textwidth}
		\centering
		\includegraphics[width=.8\linewidth]{Bilder/grayscale.png}
		\caption[Ausgabebild des Grayscalers]{Ausgabebild des Grayscalers\newline\newline\newline}
	\end{subfigure}
	\caption{Ein- und Ausgabebild des Grayscalers}
\end{figure}

	\subsection{Gauß-Filter}
	Mit dem Gauß-Filter wird das Bild wie in \autoref{sec:gaus-filter} beschrieben zur Kantenerkennung vorbereitet. Dazu bekommt die Funktion einen Zeiger auf die 8 Bit breiten Pixeldaten, die Höhe und Breite des Bildes, so wie die Kernelgröße als Argumente übergeben.
\\
\\
Mit Hilfe des Pascal'schen Dreiecks wird zunächst ein eindimensionaler Filterkernel generiert, der zuerst in horizontaler Richtung auf das Bild angewandt wird. Anschließend wird der selbe Kernel in vertikaler Richtung auf die zuvor generierten Daten angewandt. Dabei wird das Bild wie in \autoref{sec:gaus-filter} beschrieben an den Ecken um die benötigten Pixel erweitert.
\\
\\
Die For-Schleifen, in welchen der Filterkernel auf jedes Pixel angewandt wird, werden, falls \ac{OMP} verwendet wird, parallelisiert.
\\
\\
Als Ausgabe wird ein Zeiger auf die gefilterten 8 Bit breiten Daten geliefert.

	\subsection{Canny Edge Detection}
	Die Canny Edge Detection erzeugt, wie in \autoref{sec:canny-edge-detection} erläutert, ein binäres Gradientenbild, welches für die \ac{CHT} benötigt wird. Sie bekommt dabei einen Zeiger auf die 8 Bit breiten Pixeldaten, die Höhe und Breite des Bildes, so wie zwei Schwellwerte übergeben.
\\
\\
Die beiden Schwellwerte werden bei der Hysterese genutzt, um auch schwache Kanten, welche über dem niedrigeren Schwellwert liegen, aber mit starken Kanten, welche über dem höheren Schwellwert liegen, verbunden sind, im Kantenbild zu behalten.
\\
\\
Als Ausgabe wird ein Zeiger auf die 8 Bit breiten Kantendaten geliefert. Es muss von der Main Methode dafür gesorgt werden, dass die erzeugten Pixeldaten auch in ein geeignetes Bild geladen werden. Theoretisch würde an dieser Stelle auch ein einziges Bit reichen, um die Daten zu speichern, allerdings ist der kleinste Datentyp in C bereits 8 Bit groß.

\begin{figure}[htb]
	\centering
	\begin{subfigure}{.5\textwidth}
		\centering
		\includegraphics[width=.8\linewidth]{Bilder/gauss.png}
		\caption{Eingabebild der Canny Edge Detection}
	\end{subfigure}%
	\begin{subfigure}{.5\textwidth}
		\centering
		\includegraphics[width=.8\linewidth]{Bilder/canny.png}
		\caption{Ausgabebild der Canny Edge Detection}
	\end{subfigure}
	\caption{Ein- und Ausgabebild der Canny Edge Detection}
\end{figure}

	\subsection{Circle Hough Transformation}
	Die \ac{CHT} nutzt zur Erkennung von Kreisen, wie in \autoref{sec:circle-hough-transformation} erläutert, einen Akkumulatorraum und untersucht diesen anschließend nach Maxima, um Kreise zu finden.
\\
\\
Als Datenstruktur für Kreise wurde das in \autoref{lst:circle} gezeigte \emph{struct} gewählt, welches einen Kreis durch seinen Radius $r$ und die Position des Mittelpunktes $x$ und $y$ beschreibt.

\begin{lstlisting}[label=lst:circle,caption=Auszug aus hough.h]
typedef struct{
	unsigned int x;
	unsigned int y;
	uint8_t r;
} circle;
\end{lstlisting}

Zunächst wird in einem Votingverfahren jedes Kantenpixel des Bildes durchlaufen. Für jeden Radius wird mit Hilfe des \emph{Bresenham Algorithmus} ein Kreis generiert. Alle auf dem Kreis liegenden Zellen im Akkumulatorraum werden nun inkrementiert. Anschließend wird der Akkumulatorraum wie in \autoref{lst:hough} nach Maxima durchsucht.
\\
\\
In Z. \ref{lst:hough_if} wird der im Akkumulatorraum gespeicherte Wert durch die maximal mögliche Anzahl an Kreispixeln für einen vom \emph{Bresenham Algorithmus} erzeugten Kreis geteilt und anschließend mit 100 multipliziert. Hierdurch erhält man den prozentualen Anteil von den Punkten, die auf einem Kreis mit dem Radius $r$ liegen. Mit dem Schwellwert $threshold$ wird dem Programm damit ein prozentualer Parameter übergeben werden.
\\
Anschließend wird der gefundene Kreis in einem \emph{Array} vom Datentyp \emph{circle} abgelegt und der Akkumulatorraum in der näheren Umgebung auf 0 gesetzt, um Mehrfacherkennungen zu verhindern. Das \emph{Array} mit allen gefundenen Kreisen wird schließlich von der Funktion an die Main Methode zurückgegeben.

\begin{lstlisting}[label=lst:hough,caption=Auszug aus hough.c]
for(int y = 0; y < height; y++) {
        for(int x = 0; x < width; x++) {
            for(int r = radiiCount-1; r >= 0; r--) {
                int currentRadius = r+radius;
                if(acc[y * width * radiiCount + x * radiiCount + r] / (4*round((currentRadius)*SQRT2)) * 100 > threshold) {(@\label{lst:hough_if}@)
                	// Add circle
                	...
                	// Clear circle from hough space
                	...
                }
            }
        }
    }
}
\end{lstlisting}

\begin{figure}[H]
	\centering
	\begin{subfigure}{.5\textwidth}
		\centering
		\includegraphics[width=.8\linewidth]{Bilder/canny.png}
		\caption[Eingabebild der Circle Hough Transformation]{Eingabebild der \ac{CHT}}
	\end{subfigure}%
	\begin{subfigure}{.5\textwidth}
		\centering
		\includegraphics[width=.8\linewidth]{Bilder/hough.png}
		\caption[Ausgabebild der Circle Hough Transformation]{Ausgabebild der \ac{CHT}}
	\end{subfigure}
	\caption{Ein- und Ausgabebild der Circle Hough Transformation}
\end{figure}

	\section{Synthese und Optimierungen}
	Für die Auslagerung der Komponenten auf die \ac{PL} müssen diese zunächst in Hardware umgewandelt werden. Die im vorigen Kapitel vorgestellten Softwareimplementierungen werden dazu mittels Vivado HLS zu einem \ac{RTL} synthetisiert. Anschließend wird dieses über eine C/\ac{RTL} Cosimulation verifiziert und kann schließlich als \ac{IP} in Vivado zum Hardware/Software Codesign verwendet werden. Bei der Synthese werden bereits ungenaue Daten bezüglich der Laufzeit und den benötigten Ressourcen erzeugt. Diese bieten einen ersten Anhaltspunkt für die weitere Optimierung.
\\
\\
Die Optimierung findet in zwei Schritten statt: Zuerst wird die Laufzeit minimiert. Dies geschieht vor allem durch Pipelining von Schleifen oder Funktionen und dem Loop unrolling. Das Loop unrolling bezeichnet das zusammenfassen von allen Iterationsschritten einer Schleife zu einem einzigen Schritt. Dies ist oftmals für das Pipelining einer verschachtelten Schleife eine notwendige Bedingung. Jedoch führt das Loop unrolling auch zu einem hohen Ressourcenverbrauch und der Kompromiss zwischen Laufzeitersparnis und erhöhtem Hardwareaufwand muss in jedem Fall einzeln betrachtet werden.
\cite[S. 188-209]{vivado902ug}
\\
\\
Wenn die Laufzeit zufriedenstellend optimiert wurde und keine weiteren Verbesserungen mehr erzielt werden können, kann das Design anschließend hinsichtlich des Ressourcenverbrauchs optimiert werden. Hierbei können vor allem durch Datenflussdirektiven, oder das Zusammenfassen von mehreren kleineren Feldern zu einem großen (Array Mapping) Verbesserungen erzielt werden.
\cite[S. 210-219]{vivado902ug}

	\subsection{Grayscaler}\label{sec:synthese-grayscaler}
	Das Grayscaler Modul besitzt die geringste Laufzeit von allen Modulen und ist daher aus praktischer Sicht keine gute Wahl für eine \ac{HLS}. Die Synthese des Grayscaler Moduls gestaltet sich jedoch vergleichsweise sehr einfach und dient daher als gutes Beispiel, um die grundlegenden Methoden der \ac{HLS} zu zeigen.
\\
\\
Als Interfaces für den Grayscaler werden das 32 Bit breite Eingabe- und das 8 Bit breite Ausgabefeld definiert. Als einzige Anpassung am Code empfiehlt es sich, die Berechnung des Intensitätswertes auf eine ganzzahlbasierte Berechnung zu verändern.

\begin{equation}
output[index] = (30*r + 59*g + 11*b)/100
\end{equation}

Hierdurch verringert sich zwar die Präzision minimal, allerdings beträgt die Abweichung des Intensitätswertes höchstens 1 im Vergleich zur alten Berechnungsmethode. Dies ist in dem erzeugten Bild nicht erkennbar und spielt für die weitere Verarbeitung keine Rolle. Entscheidend ist, wie in \autoref{img:synthese_grayscaler400x400} zu sehen, dass der Ressourcenverbrauch von 2.837 benötigten \acp{FF} und 3.017 \acp{LUT} auf 136 benötigte \acp{FF} und 194 \acp{LUT} sinkt. Zusätzlich wird die Anzahl der benötigten Takte um einen Faktor von circa 8,2 auf 640.801 Takte reduziert.

\begin{figure}[H]
	\centering
	\includegraphics[width=0.8\linewidth]{Bilder/synthese_grayscaler400x400.png}
	\caption[Ergebnisse der Synthese des Grayscalermoduls]{Ergebnisse der Synthese des Grayscalermodul bei einem 400x400 Pixel großen Bild. Erzeugt mit Vivado HLS 2017.4 \cite{vivado902ug}}
	\label{img:synthese_grayscaler400x400}
\end{figure}

Um das Modul weiter zu Beschleunigen können im Folgenden noch die Schleifen gepipelined werden.
\\
Beim Pipelining der inneren Schleife beträgt das \ac{II} eins. Daher kann diese für ein Bild der Größe $hoehe \cdot breite$ in $hoehe \cdot breite$ Takten ausgeführt werden. Demnach kann ein Bild der Größe $400 \cdot 400$ in circa $160.000$ Takten verarbeitet werden.
\\
Um die äußere Schleife pipelinen zu können, muss die innere Schleife zunächst unrolled werden. Dies führt dazu, dass beim Pipelining $breite$-mal gleichzeitig auf das Feld $input$ zugegriffen wird. Das Feld befindet sich allerdings in einem \ac{BRAM}, welcher nur zwei Zugriffe pro Takt ermöglicht. Für das Pipelining der äußeren Schleife ist es deshalb zusätzlich nötig das Eingabe- und Ausgabefeld mit einem Faktor von $\frac{breite}{2}$ zu partitionieren. Damit auch \ac{BRAM} verwendet wird, der zwei Schreib- oder Lesezugriffe pro Takt ermöglicht, muss zusätzlich die \emph{HLS\_RESSOURCE} Direktive mit dem Parameter \emph{core=RAM\_T2P\_BRAM} angegeben werden. Beim Pipelining der äußeren Schleife hat diese schließlich auch ein \ac{II} von eins. Damit kann ein Bild in circa $hoehe$ Takten verarbeitet werden. Das entspricht $400$ Takten für ein Bild der Größe $400 \cdot 400$.
\\
\\
\autoref{img:synthese_grayscaler400x400} zeigt, dass die \ac{PL} geschätzt 160.005 Takte nach dem Pipelining der inneren Schleife benötigt. Bei einer Taktrate von 200 MHz, die als realisierbar betrachtet werden kann, wird nach der Synthese eine Laufzeit von 0,8 ms und damit ein Speedup von 5 erzielt.
\\
Nach dem Pipelining der äußeren Schleife werden geschätzt 404 Takte benötigt. Bei einer Taktrate von 200 MHz, welche auch hier als realisierbar geschätzt wird, wird eine Laufzeit von 0,002 ms und damit ein Speedup von 2000 erzielt.

	%\subsection{Gauß-Filter}
	%\input{Inhalt/HWSW_Codesign/Synthese_Gauss}
	%\subsection{Canny Edge Detection}
	%\input{Inhalt/HWSW_Codesign/Synthese_Canny}
	\subsection{Circle Hough Transformation}
	Für die Synthese des \ac{CHT} Moduls müssen verschiedene Veränderungen gegenüber der Softwareimplementierung vorgenommen werden.
\\
\\
Als Interfaces für das Modul werden das 8 Bit breite Eingabefeld mit den Daten des Eingabebildes, das Ausgabefeld, welches die gefundenen Kreise enthält und ein Zeiger auf die Anzahl der gefundenen Kreise definiert. Das Ausgabefeld ist vom Datentyp \emph{struct circle}, welcher in \autoref{lst:circle} definiert wird. %TODO: Implementierung von Zeiger in Schnittstelle erklären?
\\
\\
Zunächst muss das Feld, welches die Votingmatrix der \ac{CHT} implementiert, betrachtet werden. Dies wurde in der Softwarelösung dynamisch allokiert, um auf dem Heap gespeichert zu werden. Auf dem FPGA sollte das Akkumulatorfeld allerdings für einen effektiven Zugriff im \ac{BRAM} gespeichert werden. Für ein Bild mit einer Größe von $400 \cdot 400$ Pixeln und sieben zu durchsuchenden Radien benötigt der Akkumulator $400 \cdot 400 \cdot 7 = 1.120.000$ Elemente. In der Softwareimplementierung wurde das Feld als \emph{unsigned int} realisiert. Dieses hat auf der Zielarchitektur eine Größe von 32 Bit. Damit benötigt das Feld $32 \cdot 1.120.000 = 35.840.000$ Bit $\equiv 35,84$ Mb Speicherplatz. Die Zielplattform besitzt jedoch nur einen \ac{BRAM} von insgesamt $32,1$ Mb. Für das Beispielbild liegt die obere Grenze der zu durchsuchenden Radien bei 50. Der Bresenham Algorithmus für einen Kreis mit dem Radius von 50 erzeugt $4 \cdot \sqrt{2} \cdot 50 = 282,84 \approx 283$ Punkte. Diese können in neun Bit gespeichert werden: $2^9 = 512$. Über die \emph{Arbirary Precision Data Types Library} von Vivado kann das Feld mit einem neun Bit großem Datentyp realisiert werden. Der benötigte \ac{BRAM} beträgt dann: $9 \cdot 1.120.000 = 10.080.000$ Bit $\equiv 10,08$ Mb.
\\
Falls der \ac{BRAM} für größere Bilder nicht mehr ausreicht, kann der Akkumulator im \ac{RAM}, welcher dem FPGA zur Verfügung steht, gespeichert werden und über Linebuffer nur für die benötigten Zeilen im \ac{BRAM} gecached werden.
\\
\\
Zur weiteren Beschleunigung muss geschaut werden, an welchen Stellen Pipelining angewandt werden kann. Es gibt zwei große verschachtelte Schleifenblöcke.
\\
Der erste Block besteht aus drei verschachtelten For-Schleifen. Innerhalb der innersten For-Schleife ist noch eine While-Schleife enthalten. Dieser Block implementiert das Votingverfahren. Hier kann nur die innerste Schleife, die While-Schleife, gepipelined werden. Sie hat eine variable Anzahl an Iterationen und kann daher nicht unrolled werden. Für das Pipelining einer der äußeren Schleifen wäre dies jedoch notwendig. Innerhalb der While-Schleife wird acht mal hintereinander die \emph{vote} Methode aufgerufen. Dort wird das Akkumulatorfeld an den entsprechenden Stellen erhöht, sofern sie innerhalb der Bildgrenzen liegen. Vivado HLS erkennt nicht, dass auf verschiedene Indizes des Akkumulators zugegriffen wird und vermutet an dieser Stelle Datenabhängigkeiten. Mit der \emph{DEPENDENCE} Direktive können diese Abhängigkeiten als falsch gekennzeichnet werden. Damit kann das \ac{II} auf eins gebracht werden.
% TODO: acc muss partitioniert werden, um Anzahl der Zugriffe zu ermöglichen
\\
Der zweite Block ist in \autoref{lst:hough} dargestellt. Er filtert die höchsten Werte aus dem Akkumulator und findet dadurch die Kreise. Wenn ein Kreis gefunden wurde, muss der Akkumulator in der näheren Umgebung für alle Radien auf Null gesetzt werden. Diese Funktion wird durch drei For-Schleifen implementiert, welche in \autoref{lst:hough} an der Stelle von Z. 9 stehen.
% TODO: warum kann outer_clearing_loop trotz variablen bounds gepipelined werden?
\\
\\
Es stehen der \ac{PL} insgesamt 1.824 \acp{BRAM} zur Verfügung, von denen mit der oben beschriebenen Lösung nur $\frac{10.080 \; \text{Kb}}{18 \; \text{Kb}} = 560$ verwendet werden. %TODO: -> Partitionierung von acc für besseres II beim Pipelining

	\section{Auslagerung einzelner Komponenten auf den FPGA}
	In diesem Kapitel wird die Implementierung des Hardware/Software Codesigns beschrieben. Hierfür wird die aus der \ac{HLS} generierte \ac{IP} verwendet und mit dem Zynq UltraScale+ MPSoC verbunden. Die Kommunikation findet über das \ac{AXI4} statt. Mit dem \ac{AXI4} wird für die Kommunikation die C-API bereitgestellt, um das jeweils synthetisierte Modul anzusteuern. Eine Übersicht über die von der API bereitgestellten Methoden findet sich in \cite[S. 172f]{vivado902ug}.

	\subsection{Grayscaler}
	Die Implementierung des Grayscalers beschränkt sich auf die Verwendung von vier \acp{IP}. Die Zynq UltraScale+ MPSoC \ac{IP} \cite{vivado201pg} enthält das \ac{PS} und die \ac{PL}. Die \ac{AXI4} Interconnect \ac{IP} \cite{vivado059pg} verbindet das Grayscalermodul, welches in der \ac{PL} realisiert wird, über einen Bus mit dem \ac{PS}. Als letzte Komponente wird ein Resetmodul \cite{vivado164pg} benötigt, welches das Reset-Signal für den Grayscaler erzeugt.
\\
\\
Das \ac{PS} kann über das von dem \ac{AXI4} Modul generierte \ac{API} angesteuert werden.

\begin{figure}[htb]
	\centering
	\includegraphics[width=0.8\linewidth]{Bilder/implementierung_blockschaltbild_grayscaler400x400.png}
	\caption[Blockschaltbild für die Implementierung des Grayscalermoduls]{Blockschaltbild für die Implementierung des Grayscalermodul bei einem 400x400 Pixel großen Bild. Erzeugt mit Vivado 2017.4 \cite{vivado902ug}}
	\label{img:implementierung_timing_grayscaler400x400}
\end{figure}

Aus \autoref{img:implementierung_timing_grayscaler400x400} und \autoref{img:implementierung_utilization_grayscaler400x400} geht hervor, dass die von der Synthese in \autoref{sec:synthese-grayscaler} berechneten Werte auch nach der Implementierung eingehalten werden können. Sowohl der Platzbedarf als auch das Timing liegen in einem zufriedenstellenden Bereich.

\begin{figure}[htb]
	\centering
	\includegraphics[width=\linewidth]{Bilder/implementierung_timing_grayscaler400x400.png}
	\caption[Timing der Implementierung des Grayscalermoduls]{Timing der Implementierung des Grayscalermodul bei einem 400x400 Pixel großen Bild. Erzeugt mit Vivado 2017.4 \cite{vivado902ug}}
	\label{img:implementierung_timing_grayscaler400x400}
\end{figure}

\begin{figure}[htb]
	\centering
	\includegraphics[width=\linewidth]{Bilder/implementierung_utilization_grayscaler400x400.png}
	\caption[Ressourcenaufwand der Implementierung des Grayscalermoduls]{Ressourcenaufwand der Implementierung des Grayscalermodul bei einem 400x400 Pixel großen Bild. Erzeugt mit Vivado 2017.4 \cite{vivado902ug}}
	\label{img:implementierung_utilization_grayscaler400x400}
\end{figure}

	\chapter{Evaluation}
	Zur Evaluierung der Plattform wurden mehrere Messdaten aufgenommen. Die \ac{HT} und ihre Auslagerung in Hardware wurde mit vier verschiedenen Bildgrößen untersucht. Im Rahmen dieser Arbeit konnten sowohl Softwareimplementierung als auch Umsetzung auf der Hardware nicht bis ins letzte Detail optimiert werden. Dieses Kapitel fasst die Stärken und Schwächen des Hardware/Software Codesigns auf dem Zynq UltraScale+ \ac{MPSoC} zusammen.
\\
\\
Bisher wurden nur die Laufzeiten der einzelnen Module mit den Laufzeiten der Softwareimplementierung verglichen. Um realistischere Werte zu erhalten, muss die benötigte Zeit, um die Daten vom \ac{RAM} in den \ac{FPGA} zu laden, berücksichtigt werden. Die theoretisch maximale Performance des Speicherinterfaces der Zielplattform beträgt 2.666 Mb/s \cite{vivado890ds}. Auch, wenn dieser Wert in der Praxis nicht erreicht wird, bietet er einen ersten Anhaltspunkt, um den Vergleich realistischer zu gestalten.
\\
\\
An dieser Stelle muss noch auf eine wichtige Veränderung am Design eingegangen werden. Der Speicherbedarf der \ac{HT} ist so hoch, dass das Akkumulatorfeld für ein Bild der Größe von $800 \cdot 800$ nicht mehr in den \ac{BRAM} des \ac{FPGA} passt. Daher wurde zur besseren Auswertung für alle Bildgrößen ein zweidimensionaler \emph{Memory Window Buffer} \cite[S. 281 - 286]{vivado902ug} implementiert. Dieser reduziert den Speicherbedarf enorm und wurde so gewählt, dass alle zur parallelen Ausführung benötigten Daten innerhalb des Pufferfensters liegen und die zusätzlichen Ladezeiten kaschiert werden. Die Zeit zum einmaligen Füllen des Puffers ist so gering, dass sie vernachlässigt werden kann.

\begin{table}[!ht]
	\centering
	\caption[Ergebnisse des Hardware/Software Codesigns für verschiedene Bildgrößen]{Ergebnisse des Hardware/Software Codesigns für das \ac{CHT} Modul}
	\resizebox{\textwidth}{!}{
		\begin{tabular}{lrrrr}
			\toprule
			& \textbf{200x200} & \textbf{400x400} & \textbf{800x800} & \textbf{1.200x1.200} \\
			\toprule
			Laufzeit $[$ms$]$ & 32,00 & 123,85 & 1.080,42 & 4.172,64 \\
			Benötigte Ladezeit $[$ms$]$ & 0,60 & 4,26 & 33,13 & 106,98 \\
			Laufzeit + Ladezeit $[$ms$]$ & 32,60 & 128,11 & 1.113,55 & 4.279,62 \\
			Laufzeit Software $[$ms$]$ & 94,00 & 1.044,00 & 24.759,00 & 274.957,00 \\
			\textbf{Speedup} & 2,88 & 8,15 & \textbf{22,23} & \textbf{64,25} \\
			FPS & \textbf{30,67} & 7,81 & 0,9 & 0,23 \\
			\hline
			BRAM & 40 (2,19\%) & 141 (7,73\%) & \textbf{545 (29,88\%)} & \textbf{1.091 (59,81\%)} \\
			DSP48E & 19 (0,75\%) & 19 (0,75\%) & 32 (1,27\%) & 32 (1,27\%) \\
			FF & 3.233 (0,59\%) & 3.911 (0,71\%) & 5.125 (0,93\%) & 9.815 (1,79\%) \\
			LUT & 7.477 (2,73\%) & 8.764 (3,20\%) & 9.425 (3,44\%) & 12.054 (4,40\%) \\
			\textbf{FPGA-Ressourcen} & \textbf{1,57\%} & \textbf{3,10\%} & \textbf{8,88\%} & \textbf{16,82\%} \\
			\textbf{Speedup $\div$ FPGA-Ress.} & 183,44 & 262,90 & \textbf{250,34} & \textbf{381,99} \\
			\bottomrule
		\end{tabular}
	}
	\label{tab:evaluation_hwsw_codesign}
\end{table}

Bei Betrachtung der Ergebnisse aus \autoref{tab:evaluation_hwsw_codesign} fällt auf, dass der erzielte Speedup mit wachsender Bildgröße deutlich ansteigt. Das Verhältnis zwischen Speedup und benötigten FPGA-Ressourcen ist für das größte Bild am besten und für das kleinste Bild am schlechtesten. Es kann für größere Bilder also mit weniger zusätzlichem Ressourcenaufwand ein höherer Speedup gegenüber der Softwareimplementierung erzielt werden. Das Hardware/Software Codesign der \ac{HT} lohnt sich für die betrachteten Größen umso mehr, je größer die Bilder sind. Es liegen keine Ergebnisse zur Laufzeit von Bildern vor, für welche Eingabefeld und Puffer nicht mehr in den \ac{BRAM} passen. Daher kann keine konkrete Aussage über den Leistungsabfall für größere Bilder getroffen werden. Nachfolgend werden deshalb nur Bildgrößen von maximal $5.793 \cdot 5.793$ Pixel, also 33,56 Megapixel, diskutiert.
\\
\\
Von den vier Testbildern erreicht nur die \ac{HT} des kleinsten Bildes ausreichend \ac{fps}, um in Echtzeit auf ein Videosignal angewandt werden zu können. Ab einer Bildgröße von $800 \cdot 800$ Pixel kann nicht mal mehr ein Bild pro Sekunde verarbeitet werden. Zur Erkennung von Objekten im Straßenverkehr eignet sich die \ac{HT} daher selbst mit Hardwarebeschleunigung nicht.
\\
\\
Durch die vielen von der \ac{HT} benötigten Parameter kann das Verfahren nur schlecht unter wechselnden Bedingungen, wie bei sich verändernden Lichtverhältnissen, genutzt werden. Mit verschiedenen Kernelgrößen für das Gauß-Filter und verschiedenen Schwellwerten für die Canny Edge Detection werden unterschiedlich gute Ergebnisse erzielt. Die Parameter der \ac{HT} müssen vor der Synthese, z.B. durch a priori Wissen, bestimmt werden. Wenn diese einmal in der \ac{PL} implementiert sind, lassen sie sich erst über eine erneute Synthese wieder verändern. Dadurch ist das Design nicht so flexibel wie die Softwarelösung.
\\
Das Hardware/Software Codesign der \ac{HT} eignet sich deshalb auf der Zielplattform nur für einen spezifischen Anwendungsfall. Bei einer festen Bildgröße und festen Parametern kann das Verfahren für diese optimiert werden. Bei wechselnden Bildgrößen oder Anwendungen mit variablen Parametern eignet sich die Implementierung der \ac{HT} in Hardware nicht.
\\
\\
In dem Beispielbild wird jedoch trotz der geringen Ressourcennutzung für alle Bildgrößen ein sehr guter Speedup gegenüber der reinen Softwareimplementierung erzielt. Der Zynq Ultrascale+ MPSoC eignet sich gut, um Teile eines Algorithmus, hier insbesondere das kostenintensive Votingverfahren der \ac{CHT}, auszulagern und zu beschleunigen.

	\chapter{Fazit und Ausblick}
	Ziel der Bachelorarbeit war der Entwurf und die Evaluation eines Hardware/Software Codesigns am Beispiel einer \ac{HT}. Zu diesem Zweck wurde zunächst eine Software für die \ac{HT} zur Erkennung von Kreisen entwickelt. Diese wurde in verschiedene Module aufgeteilt, von denen zwei mittels \ac{HLS} synthetisiert wurden. Anschließend wurde die Laufzeit der Software auf der Zielplattform gemessen und es wurde eine mögliche Auslagerung der Module auf den rekonfigurierbaren Bereich der Zielplattform aufgezeigt. Schließlich wurden die synthetisierten Module hinsichtlich Laufzeit und Ressourcenverbrauch untersucht und mit der Softwarelösung verglichen.
\\
Dabei ergab sich, dass die Beschleunigung der \ac{HT} mittels Hardware/Software Codesign insgesamt gut funktioniert. Das Hardware/Software Codesign hat für alle getesteten Bildgrößen einen Speedup erzielt, aber sich vor allem für die größeren Testbilder als effektiv herausgestellt. Das erzeugte Codesign war für das $1.200 \cdot 1.200$ Pixel große Testbild in der Lage, die \ac{HT} um einen Faktor von etwa 64 zu beschleunigen.
\\
Andererseits musste festgestellt werden, dass die Beschleunigung bei größeren Bildern als $400 \cdot 400$ Pixel nicht mehr ausreicht, um eine Kreiserkennung in Echtzeit realisieren zu können. Die \ac{HT} benötigt außerdem viele Parameter, welche nach der Synthese fest in die Hardwarelösung implementiert sind. Damit ist sie nicht mehr flexibel und kann z.B. nicht auf Bilder mit unterschiedlichen Lichtverhältnissen angewandt werden. Für eine Veränderung der Parameter muss eine erneute Synthese durchgeführt werden.
\\
\\
Im Verlauf der Arbeit haben sich weitere Optimierungen aufgezeigt, welche im Rahmen dieser Bachelorarbeit nicht mehr betrachtet werden konnten.
\\
Die \ac{HLS} wurde nur für das Grayscaler und \ac{CHT} Modul durchgeführt. Auf Grund der viel höheren Laufzeit ist ein Fokus auf die Optimierung des \ac{CHT}-Moduls sinnvoll gewesen. Nicht genutzte Ressourcen können von den anderen, in dieser Arbeit nicht synthetisierten Modulen, verwendet werden. Es bietet sich auch die Möglichkeit des Pipelinings gesamter Module an. Die Module können dann über die \ac{PS} angesteuert werden und ineinander verzahnt laufen. Damit werden mehrere Bilder gleichzeitig verarbeitet und folglich auch die \ac{fps} ansteigen.
\\
Des Weiteren können \ac{BRAM}-Blöcke eingespart werden, indem die Bittiefe des von der Canny Edge Detection erzeugten binären Gradientenbildes verringert wird. Da dieses Bild nur zwei Farben (Schwarz und Weiß) enthält, ist eine Bittiefe von eins ausreichend. Im aktuellen Design wird, wie in der Softwareimplementierung, eine Bittiefe von acht Bit benutzt. Dies verringert die Größe das Eingangsfeld des \ac{CHT} Modules um einen Faktor von acht.
\\
Außerdem lässt sich für die Synthese des \ac{CHT} Moduls der Code so umschreiben, dass ein effektiveres Pipelining durchgeführt werden kann. Das Votingverfahren beansprucht den Großteil der Laufzeit. Dieses kann in eine Funktion ausgelagert werden und für alle Radien gleichzeitig ausgeführt werden.
\\
Schließlich können die synthetisierten Module nicht ohne Weiteres auf einem Betriebssystem genutzt werden, da dort der Kernel die Ansteuerung der Hardware übernimmt. Für eine Realisierung des Codesigns unter einem laufenden Betriebssystem müssen die von der \ac{HLS} automatisch generierten Treiber angepasst werden. Durch die Zwischenschaltung vom Kernel ist das Design dann zwar deutlich langsamer als die direkt auf das \ac{PS} und die \ac{PL} aufgespielte Anwendung, aber dennoch von Interesse. So können z.B. die Vorzüge eines Dateisystems genutzt und damit das Laden verschiedener Bilder ermöglicht werden..
\\
\\
Das Hardware/Software Codesign der \ac{HT} ist eine vielseitige Aufgabe gewesen, welche am Ende ein zufriedenstellendes Resultat erbracht hat. Eine Fragestellung, die am Ende für die Zukunft offen bleibt, ist, ob eine \ac{HT} auch für ein hochauflösendes Bild auf Echtzeit beschleunigt werden kann.

	
	%---Leerseite--------->
	\newpage
	\mbox{}
	\thispagestyle{empty}
	\newpage
	%--------------------->	
	
	\setstretch{1,1}
	
	\pagenumbering{roman}
	%\setcounter{page}{1}
	\addcontentsline{toc}{chapter}{Literaturverzeichnis}
	\bibliography{Inhalt/Literatur}
	
	%\include{Inhalt/Anhang/Anhang}

\end{document}
