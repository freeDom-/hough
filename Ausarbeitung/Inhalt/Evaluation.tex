Zur Evaluierung der Plattform wurden mehrere Messdaten aufgenommen. Die \ac{HT} und ihre Auslagerung in Hardware wurde mit vier verschiedenen Bildgrößen untersucht. Im Rahmen dieser Arbeit konnten sowohl Softwareimplementierung wie auch Umsetzung auf der Hardware nicht bis ins letzte Detail optimiert werden. Dieses Kapitel zeigt allerdings die Stärken und Schwächen des Zynq UltraScale+ \ac{MPSoC} beim Hardware/Software Codesign einer \ac{HT}.
\\
\\
Bla

\begin{table}[H]
	\centering
	\caption[Ergebnisse der High Level Synthese für verschiedene Bildgrößen]{Ergebnisse der High Level Synthese für das \ac{CHT} Modul. In den Spalten stehen die verschiedenen Optimierungsschritte. Die prozentualen Angaben sind auf ganze Prozent gerundet.}
	\resizebox{\textwidth}{!}{
		\begin{tabular}{lrrrr}
			\toprule
			& \textbf{200x200} & \textbf{400x400} & \textbf{800x800} & \textbf{1.200x1.200} \\
			\toprule
			Zieltakt $[$ns$]$ & 5 & 5 & 5 & 5 \\
			Erw. Takt $[$ns$]$ & 4,69 &  &  &  \\
			Latenz $[$Takte$]$ & 6.961.472 &  &  &  \\
			Laufzeit Software $[$ms$]$ & 94 &  &  &  \\
			Laufzeit $[$ms$]$ & 32 &  &  &  \\
			\textbf{Speedup} & 2,94 &  &  &  \\
			\hline
			BRAM & 118 (6\%) &  (\%) &  (\%) &  (\%) \\
			DSP48E & 19 (1\%) &  (\%) &  (\%) &  (\%) \\
			FF & 3.233 (1\%) &  (\%) &  (\%) &  (1\%) \\
			LUT & 7.477 (3\%) &  (\%) &  (\%) &  (\%) \\
			\textbf{FPGA-Ressourcen} & 3\% & \% & \% & \% \\
			\textbf{Speedup $\div$ FPGA-Ress.} & 98 &  &  &  \\
			\bottomrule
		\end{tabular}
	}
	\label{tab:synthese_hough400x400}
\end{table}
