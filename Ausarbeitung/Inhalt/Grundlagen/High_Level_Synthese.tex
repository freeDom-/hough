In den vergangenen Jahrzehnten ist die Komplexität von Hardware und integrierten Schaltungen immer weiter angestiegen. Daher wurde es notwendig, neue Entwurfsmethoden auf einem höheren Abstraktionslevel einzuführen, um eine einfachere Entwicklung zu ermöglichen. Die \ac{HLS} ist ein automatisierter Entwurfsprozess, der aus Programmcode digitale Hardware generiert. Diese erfüllt die Funktionalität des Codes.
\\
In der Softwarebranche wurden ähnliche Schritte mit der Ablösung von Maschinencode durch Assembler und später durch Hochsprachen wie C durchgeführt. Auf Grund der Komplexität heutiger Software ist es nahezu undenkbar, diese vollständig in Assembler zu entwickeln.
\cite[S. 8]{coussy2009hls}
\\
\\
Der erste Schritt einer \ac{HLS} besteht aus der Kompilierung des Codes zu Kontroll- und Datenflussgraphen. Dabei werden zumeist auch verschiedene Optimierungen vorgenommen. Darunter fallen unter anderem die Eliminierung von unerreichbarem Code, die Substitution der Ergebnisse von Berechnungen durch Konstanten und Schleifenoptimierungen durchgeführt.
\\
Anschließend werden Typ und Anzahl der benötigten Hardware Ressourcen bestimmt. Diese sind z.B. Speichereinheiten, Busse oder funktionale Einheiten. Sie werden aus einer \ac{RTL} Bibliothek ausgewählt. Darin sind verschiedene Komponenten sowie deren Spezifikationen (Größe, Delay, benötigte Energie) beinhaltet.
\\
Im nächsten Schritt findet das \emph{Scheduling} von verschiedenen Operationen des Typs $a = b \odot c$ statt. Dabei steht $\odot$ für einen beliebigen Operator. Die Variablen $b$ und $c$ müssen entweder aus einem Speicher oder aus der Ausgabe einer funktionalen Einheit geladen werden. Operationen können verkettet oder parallel ausgeführt werden, falls es keine Datenabhängigkeiten gibt.
\\
Die benötigten Variablen müssen an Speichereinheiten gebunden werden. Sofern sich die \emph{Lebenszeiten} von verschiedenen Variablen nicht überlappen, können diese an die selbe Speichereinheit gebunden werden. Operationen müssen außerdem an funktionale Einheiten gebunden werden, welche in der Lage sind, die Operation auszuführen. Wenn es mehrere solcher Einheiten gibt, optimiert der \emph{Binding Algorithmus} die Auswahl. Mit dem Binden von funktionalen Einheiten und Speichereinheiten werden diese gleichzeitig an Verbindungseinheiten wie Busse gebunden.
\\
Im letzten Schritt der \ac{HLS} wird schließlich ein \ac{RTL} Modell des synthetisierten Designs generiert.
\cite[S. 9-11]{coussy2009hls}
