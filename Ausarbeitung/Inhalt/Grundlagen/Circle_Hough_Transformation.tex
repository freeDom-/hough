Die \ac{CHT} ist eine Variation der \ac{HT} zur Erkennung von Kreisen in Bildern. Dabei werden als Parameter zur eindeutigen Beschreibung des Kreises die Koordinaten des Mittelpunktes $(x_0,y_0)$ und der Radius $r$ verwendet. Als Eingabe benötigt der Algorithmus ein binäres Gradientenbild, welches z.B. durch eine vorherige Kantenerkennung erzeugt wurde.
\\
\\
Ein Punkt $p = (x,y)$ liegt auf einem Kreis mit den gegebenen Parametern, wenn die folgende Gleichung erfüllt ist.

\begin{equation}
(x-x_0)^2+(y-y_0)^2 = r^2
\end{equation}

Die \ac{CHT} benötigt deshalb einen dreidimensionalen Parameterraum $A(x_0,y_0,r)$, um Position und Radius von Kreisen in Bildern zu finden. Da keine einfache funktionale Abhängigkeit zwischen den Koordinaten im Parameterraum existiert, ist ein anderer Weg notwendig, um alle Parameterkombinationen $(x_0,y_0,r)$ zu finden. Eine Möglichkeit wäre, für jedes Element im Parameterraum zu testen, ob es die Gleichung erfüllt. Dieser \emph{brute force} Ansatz ist jedoch sehr rechenintensiv. Es kann sich auch die Eigenschaft aus \autoref{img:circle-hough-transformation} zunutze gemacht werden. Hier sieht man, dass die Koordinaten auch im Hough Raum einen Kreis beschreiben und jeder Kantenpunkt des Kreises $C$ einen Kreis mit dem selben Radius $\rho_i$ hat, welcher wieder durch den Kreismittelpunkt geht. Es ist daher nicht notwendig, den gesamten dreidimensionalen Parameterraum für jeden Bildpunkt zu testen. Stattdessen wird an jeder Kante für alle möglichen Radien $r$ mit Hilfe eines Standardalgorithmus wie dem Bresenham Algorithmus ein Kreis generiert. Alle auf dem Kreis liegenden Zellen werden im Akkumulator $A(x,y,r)$ entsprechend inkrementiert. Zuletzt wird der Akkumulatorraum nach den höchsten Einträgen durchsucht. Dies sind die Kreise, welche die meisten Kanten enthalten.
\cite[S. 64]{burger2009podip_2}

\begin{figure}[H]
	\centering
	\includegraphics[width=0.8\linewidth]{Bilder/circle_hough_transform-burger2009podip_2.pdf}
	\caption[Circle Hough Transformation]{Circle Hough Transformation\\\cite[S. 65]{burger2009podip_2}}
	\label{img:circle-hough-transformation}
\end{figure}

Der Akkumulatorraum wird nur nach den höchsten Einträgen durchsucht. Deshalb besteht das Problem, dass kleinere Kreise nicht mehr erkannt werden, weil diese kleinere Werte im Akkumulatorraum haben als große Kreise. Daher ist die Durchführung einer Biaskompensation notwendig. Hierfür wird jeder Wert im Akkumulatorraum normalisiert, indem er durch die maximal mögliche Anzahl an Kantenpixeln geteilt wird.
\cite[S. 171f]{burger_2016dip_java}
