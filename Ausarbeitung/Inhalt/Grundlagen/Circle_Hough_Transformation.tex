Die \ac{CHT} ist eine Variation der \ac{HT} zur Erkennung von Kreisen in Bildern. Dabei werden als Parameter zur eindeutigen Beschreibung des Kreises die Koordinaten des Mittelpunktes $(x_0,y_0)$ und der Radius $r$ verwendet. Als Eingabe benötigt der Algorithmus ein binäres Gradientenbild, welches z.B. durch eine vorherige Kantenerkennung erzeugt wurde.
\\
\\
Ein Punkt $p = (x,y)$ liegt auf einem Kreis mit den gegebenen Parametern, wenn die folgende Gleichung erfüllt ist.
\\
\begin{equation}
(x-x_0)^2+(y-y_0)^2 = r^2
\end{equation}
\\
Daher benötigt die \ac{CHT} einen dreidimensionalen Parameterraum $A(x_0,y_0,r)$, um Position und Radius von Kreisen in Bildern zu finden. Es existiert keine einfache funktionale Abhängigkeit zwischen den Koordinaten im Parameterraum.
Voting erklären usw....
\cite[S. 64]{burger2009podip_2}
