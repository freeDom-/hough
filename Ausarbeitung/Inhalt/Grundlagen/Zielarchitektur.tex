Als Zielarchitektur für diese Arbeit wird das Xilinx Zynq UltraScale+ MPSoC ZCU102 Evaluation Board verwendet. Es besitzt ein \ac{MPSoC} mit einer \ac{PL} und einem \ac{PS}, bestehend aus mehreren Prozessoren.
\\
\\
Der rekonfigurierbare Teil besteht aus einem Zynq UltraScale XCZU9EG-2FFVB1156 FPGA mit 600 logischen Einheiten, 32.1 Mb Speicher, 2.520 \acp{DSP} und 328 I/O Pins. Dem FPGA stehen 512 MB DDR4 \ac{RAM} bei 1200 MHz / 2400 Mbps zur Verfügung.
\\
\\
Das \ac{PS} beinhaltet eine \ac{APU}, zwei \acp{RPU} und eine \ac{GPU}. Es befinden sich ein Cortex-A53 64-bit Vierkernprozessor mit jeweils 32 KB L1 Cache und 1 MB L2 Cache auf dem \ac{MPSoC}, sowie zwei Cortex-R5 \acp{RPU}. Außerdem verfügt das Board über eine Mali-400 MP2 \ac{GPU} mit 64 KB L2 Cache. Den Prozessoren stehen 4 GB DDR4 \ac{RAM} zur Verfügung.
\cite{zcu102ug}

\begin{figure}[htb]
	\centering
	\includegraphics[width=\linewidth]{Bilder/zynq-zcu102ug.pdf}
	\caption[Zynq Ultrascale+ MPSoC Block Diagramm]{ZynQ Ultrascale+ MPSoC Bock Diagramm\\Quelle: \cite[S. 21]{zcu102ug}}
\end{figure}
