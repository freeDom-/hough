Damit eine Kantenextraktion mit dem Canny-Algorithmus (\autoref{sec:canny-edge-detection}) durchgeführt und somit ein binäres Gradientenbild erzeugt werden kann, muss zuvor eine Umwandlung eines Farbbildes in ein Graustufenbild erfolgen. Ein Bild ist grau, wenn die Rot-, Grün- und Blaukomponenten $(R, G, B)$ den selben Wert haben. Dann kann jedoch anstelle der zuvor benötigten drei Farbkanäle $(R, G, B)$ mit je 8 Bit pro Pixel nur noch ein Farbkanal mit 8 Bit pro Pixel verwendet werden. Hierbei wird die Größe des Bildes und damit auch die Größe der zu verarbeitenden Daten deutlich reduziert.
\\
\\
Für jedes Pixel werden die drei Farbkomponenten $(R, G, B)$ in einen Intensitätswert $Y$ umgerechnet. Dieser beschreibt die Helligkeit eines Pixels. Es gibt verschiedene Ansätze die Intensität zu berechnen. An dieser Stelle soll nicht näher auf die Unterschiede eingegangen werden. Zwei detaillierte Vergleiche zur Effektivität der verschiedenen Ansätze hinsichtlich einer späteren Kantenextraktion finden sich in \cite{kanan2012grayscale} und \cite{ahmad2018grayscale}. GLuminance gilt als Standardalgorithmus der Bildverarbeitung und wir daher im folgenden betrachtet. Die Intensität berechnet sich nach GLuminance wie in (\ref{eq:gluminance}) gewichtet, da wir rot und grün deutlich heller als blau wahrnehmen.

\begin{equation}\label{eq:gluminance}
Y = 0.3 * R + 0.59 * G + 0.11 * B
\end{equation}

Die Umwandlung eines Bildes der Größe $M \times N$ benötigt $5(M \cdot N)$ Operationen. Unter der vereinfachten Annahme, dass die Bildgröße $N \times N$ ist, folgt damit eine Komplexität von $O(N^2)$.
