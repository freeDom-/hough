Damit eine Kantenextraktion mit dem Canny-Algorithmus (\ref{canny}) durchgeführt und somit ein binäres Gradientenbild erzeugt werden kann, muss zuvor eine Umwandlung eines Farbbildes in ein Graustufenbild erfolgen. Ein Bild ist grau, wenn die rot, grün und blau $(R, G, B)$ Komponenten den selben Wert haben. Dann kann allerdings auch anstatt der zuvor benötigten drei Farbkanäle $(R, G, B)$ mit je 8 Bit pro Pixel nur noch ein Farbkanal mit 8 Bit pro Pixel verwendet werden. Damit wird die Größe des Bildes und damit auch die Größe, der zu verarbeitenden Daten deutlich reduziert.
\\
\\
Für jedes Pixel werden die drei Farbkomponenten $(R, G, B)$ in einen Intensitätswert $Y$ umgerechnet, welcher die Helligkeit des Pixels beschreibt. Es gibt eine Vielzahl verschiedener Ansätze, die sich unterschiedlich gut zur Kantenextraktion eignen. An dieser Stelle soll nicht näher auf die Unterschiede eingegangen werden. Zwei detaillierte Vergleiche zu dem Thema finden sich in \cite{kanan2012grayscale} und \cite{ahmad2018grayscale}. Daraus geht hervor, dass GLuminance als Standardalgorithmus zur Bildverarbeitung gilt. Die Intensität berechnet sich nach GLuminance gewichtet, da wir rot und grün deutlich heller als blau wahrnehmen.
\\
\begin{equation}
Y = 0.3 * R + 0.59 * G + 0.11 * B
\end{equation}
\\
Die Umwandlung eines Bildes der Größe $M \times N$ benötigt $5(M \cdot N)$ Operationen. Unter der vereinfachten Annahme, dass die Bildgröße $N \times N$ ist folgt damit eine Komplexität von $O(N^2)$.
