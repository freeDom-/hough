Mit dem Bresenham Algorithmus können die Punkte eines Kreises mit gegebenem Radius $r$ und dem Mittelpunkt $(x_0, y_0)$ berechnet werden. Da der Algorithmus nur auf ganzzahligen Berechnungen basiert, ist er besonders effektiv und zählt zu den Standardalgorithmen in der Bildbearbeitung.
\\
\\
Auf Grund der Symmetrie eines Kreises kann dieser in acht Oktanten unterteilt werden. Der Bresenham Algorithmus zeichnet bei jedem Schritt für jeden Oktanten ein Pixel.
\\
Am Anfang können außerdem bereits die Punkte $(x_0+r, y_0)$, $(x_0-r, y_0)$, $(x_0, y_0+r)$ und $(x_0, y_0-r)$ eingezeichnet werden. Im Folgenden wird der Algorithmus für den ersten Oktanten gezeigt. Bei den anderen Oktanten kann analog verfahren werden.
\\
\\
Für den ersten Oktanten zeichnet der Algorithmus in jedem Schritt ein Pixel an der Position $(x_0+x, y_0+y)$. Dabei sind x und y Laufvariablen, die sich mit jedem Schritt verändern. Es gibt eine schnelle und eine langsame Laufrichtung. Die schnelle Richtung ist im Falle des ersten Oktanten die y-Richtung. Diese wird mit 0 initialisiert und bei jedem Schritt um eins erhöht. Die langsame Richtung, also die x-Richtung, wird mit $r$ initialisiert und abhängig von einer Fehlervariablen dekrementiert. Der Fehler wird zu Beginn auf den Radius gesetzt. Bei jedem Schritt wird $y*2 + 1$ von dem Fehler abgezogen. Wenn die Fehlervariable kleiner als 0 ist wird die langsame Richtung um eins verringert und $1 - 2 \cdot x$ vom Fehler abgezogen. Der Algorithmus terminiert, wenn die schnelle Richtung größer oder gleich der langsamen Richtung ist, also wenn $y \ge x$ gilt. Die Anzahl der erzeugten Punkte für einen Kreis beträgt: $4 \cdot \sqrt{2} \cdot r$.
\cite{bresenham1977circle}

%TODO: neues Bild, eigenes Bild?
\begin{figure}[H]
	\centering
	\includegraphics[width=0.5\linewidth]{Bilder/Bresenham_circle.png}
	\caption[Bresenham Algorithmus]{Bresenham Algorithmus $[$https://upload.wikimedia.org/wikipedia/de/0/09/Bresenham\_circle2.png, 02.07.2018$]$}
	\label{img:bresenham}
\end{figure}
