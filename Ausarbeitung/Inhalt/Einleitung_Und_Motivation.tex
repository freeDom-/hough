Bis vor einigen Jahren hatten die meisten Autos ein \ac{ABS} und ein \ac{ESP} zur Unterstützung bei Gefahrenbremsungen und zur Stabilisierung der Fahrt in scharfen Kurven. Mittlerweile ist die Liste der Fahrerassistenzsysteme deutlich größer geworden und umfasst unter anderem auch adaptive Geschwindigkeitsregelanlagen, Spurhalte- und Spurwechselassistenten, Einparkhilfen und autonome Notbremssysteme \cite[S. 26]{mut2018fahrerassistenzsysteme}. Viele dieser Systeme basieren auf Bildbearbeitungs- und Bildanalysealgorithmen, welche gerade in Gefahrensituationen schnellstmöglich ausgewertet werden müssen. Hierfür ist die Nutzung einer \ac{CPU} meist keine ausreichende Lösung, da diese zwar eine vergleichsweise hohe Taktrate besitzt, aber nicht genug Parallelität bietet, um große Bilder effektiv auswerten zu können. Solche Probleme werden daher vermehrt von \acp{FPGA} und \acp{GPU} gelöst in Hardware ausgelagert \cite[S. 25]{mut2018fahrerassistenzsysteme}\cite[S. 452]{hauck2008reconfigurablecomputing}.
\\
Diese Bachelorarbeit behandelt eine solche Auslagerung von Software in Hardware. Dazu wird einleitend die Zielarchitektur vorgestellt und ein Einblick in die Grundlagen des Hardware/Software Codesigns, so wie der \ac{HLS} gegeben. Anschließend wird am Beispiel einer \acl{HT} eine \ac{HLS} durchgeführt. Dazu wird zunächst die Softwareimplementierung vorgestellt und nachfolgend iterativ einzelne Komponenten der \acl{HT} von einer \ac{CPU} auf einen \ac{FPGA} ausgelagert.
\\
Im Anschluss wird eine Auswertung des Hardware/Software Codesigns hinsichtlich des erbrachten Speedups und des Ressourcenverbrauchs durchgeführt. Es wird diskutiert, inwiefern sich die Zielarchitektur für ein solches Hardware/Software Codesign eignet und welche Grenzen es dabei gibt. Abschließend wird ein Fazit gezogen und ein Ausblick zu dem Thema gegeben.
