In diesem Kapitel wird die Softwareimplementierung der \ac{HT} vorgestellt. Da das zur anschließenden \ac{HLS} verwendete Programm \emph{Vivado HLS} für C Code ausgelegt ist wurde C als Programmiersprache gewählt. Die Implementierung kann in 5 verschiedene Module unterteilt werden, welche die unterschiedlichen Funktionalitäten implementieren. Dabei ist ein modularer Aufbau zur Übersicht, Zeiterfassung und vor allem für die spätere \ac{HLS} von großem Vorteil. Es lässt sich jedes Modul einzeln synthetisieren und so in \ac{HW} auslagern. Die einzelnen Module werden in den folgenden Unterkapiteln näher beschrieben.
\\
\\
Als Compiler wurde GCC (TODO VERSIONNR.) verwendet.
\\
Zum Laden des zu verarbeitenden Bildes und der Pixeldaten wurde die Bibliothek \ac{SDL 2}\footnote{\url{https://www.libsdl.org/}} (TODO VERSIONNR.) verwendet, welche einen leichten Zugriff auf die Pixeldaten ermöglicht und es einfach gestaltet die von den Zwischenschritten erzeugten Bilder abzuspeichern.
\\
Zur Parallelisierung des Programmes wurde \ac{OMP}\footnote{\url{https://www.openmp.org/}} (TODO VERSIONNR.) genutzt. \ac{OMP} ermöglicht eine leichte Parallelisierung durch Präprozessoranweisungen und bietet über das Compilerflag \emph{-fopenmp}, welches das Präprozessormakro \emph{\_OPENMP} definiert, eine einfache Möglichkeit, um eine serielle und parallele Version des Programms zu kompilieren.

\begin{figure}[htb]
	\centering
	\includegraphics[width=\linewidth]{Bilder/usage.png}
	\caption{Benutzeranleitung für das Programm}
	\label{img:usage}
\end{figure}

Das Programm wird in Unix mit \emph{./hough} (oder \emph{./hough\_omp} für die parallele Version) ausgeführt. Die möglichen Argumente sind \autoref{img:usage} zu entnehmen.
