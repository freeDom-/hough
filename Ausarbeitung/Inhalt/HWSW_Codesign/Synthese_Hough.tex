Für die Synthese des \ac{CHT} Moduls müssen verschiedene Veränderungen gegenüber der Softwareimplementierung vorgenommen werden.
\\
\\
Als Interfaces für das Modul werden das 8 Bit breite Eingabefeld mit den Daten des Eingabebildes, das Ausgabefeld, welches die gefundenen Kreise enthält und ein Zeiger auf die Anzahl der gefundenen Kreise definiert. Das Ausgabefeld ist vom Datentyp \emph{struct circle}, welcher in \autoref{lst:circle} definiert wird. %TODO: Implementierung von Zeiger in Schnittstelle erklären?
\\
\\
%TODO: genauer erklären: Welches Akkumulatorfeld? + Einleitung in Abschnitt?
Das Akkumulatorfeld wurde in der Softwarelösung dynamisch allokiert, um auf dem Heap gespeichert zu werden. Auf dem FPGA sollte das Akkumulatorfeld allerdings für einen effektiven Zugriff im \ac{BRAM} gespeichert werden. Für ein Bild mit einer Größe von $400 \cdot 400$ Pixeln und sieben zu durchsuchenden Radien benötigt der Akkumulator $400 \cdot 400 \cdot 7 = 1.120.000$ Elemente. In der Softwareimplementierung wurde das Feld als \emph{unsigned int} realisiert. Dieses hat auf der Zielarchitektur eine Größe von 32 Bit. Damit benötigt das Feld $32 \cdot 1.120.000 = 35.840.000$ Bit $\equiv 35,84$ Mb Speicherplatz. Die Zielplattform besitzt jedoch nur einen \ac{BRAM} von insgesamt $32,1$ Mb. Für das Beispielbild liegt die obere Grenze der zu durchsuchenden Radien bei 50. Der Bresenham Algorithmus für einen Kreis mit dem Radius von 50 erzeugt $4 \cdot \sqrt{2} \cdot 50 = 282,84 \approx 283$ Punkte. Diese können in neun Bit gespeichert werden: $2^9 = 512$. Über die \emph{Arbirary Precision Data Types Library} von Vivado kann das Feld mit einem neun Bit großem Datentyp realisiert werden. Der benötigte \ac{BRAM} beträgt dann: $9 \cdot 1.120.000 = 10.080.000$ Bit $\equiv 10,08$ Mb.
\\
Falls der \ac{BRAM} für größere Bilder nicht mehr ausreicht, kann der Akkumulator im \ac{RAM}, welcher dem FPGA zur Verfügung steht, gespeichert werden und über Linebuffer nur für die benötigten Zeilen gecached werden.
\\
\\
Zur weiteren Beschleunigung muss geschaut werden, welche Schleife gepipelined werden kann.
\\
\\
Es stehen der \ac{PL} insgesamt 1.824 \acp{BRAM} zur Verfügung, von denen mit der oben beschriebenen Lösung nur $\frac{10.080 \; \text{Kb}}{18 \; \text{Kb}} = 560$ verwendet werden.
