Die \ac{CHT} nutzt zur Erkennung von Kreisen, wie in \autoref{sec:circle-hough-transformation} erläutert, einen Akkumulatorraum und untersucht diesen anschließend nach Maxima, um Kreise zu finden.
\\
\\
Als Datenstruktur für Kreise wurde das in \autoref{lst:circle} gezeigte \emph{struct} gewählt, welches einen Kreis durch seinen Radius $r$ und die Position des Mittelpunktes $x$ und $y$ beschreibt.

\begin{lstlisting}[label=lst:circle,caption=Auszug aus hough.h]
typedef struct{
	unsigned int x;
	unsigned int y;
	uint8_t r;
} circle;
\end{lstlisting}

Zunächst wird in einem Votingverfahren jedes Kantenpixel des Bildes durchlaufen. Für jeden Radius wird mit Hilfe des \emph{Bresenham Algorithmus} ein Kreis generiert. Alle Zellen im Akkumulatorraum, die auf dem Kreis liegen werden nun inkrementiert. Anschließend wird der Akkumulatorraum wie in \autoref{lst:hough} nach Maxima durchsucht.
\\
\\
In Z. \ref{lst:hough_if} wird der im Akkumulatorraum gespeicherte Wert durch die maximal mögliche Anzahl an Kreispixeln für einen vom \emph{Bresenham Algorithmus} erzeugten Kreis geteilt und anschleßend mit 100 multipliziert. Hierdurch erhält man den prozentualen Anteil von den Punkten, die auf einem Kreis mit dem Radius $r$ liegen und mit dem Schwellwert $threshold$ kann dem Programm dann der Einfachheit halber ein prozentualer Parameter übergeben werden.
\\
Anschließend wird der gefundene Kreis in einem \emph{Array} vom Datentyp \emph{circle} abgelegt und der Akkumulatorraum in der näheren Umgebung auf 0 gesetzt, um Mehrfacherkennungen zu verhindern. Das \emph{Array} mit allen gefundenen Kreisen wird anschließend von der Funktion an die Main Methode zurückgegeben.

\begin{lstlisting}[label=lst:hough,caption=Auszug aus hough.c]
for(int y = 0; y < height; y++) {
        for(int x = 0; x < width; x++) {
            for(int r = radiiCount-1; r >= 0; r--) {
                int currentRadius = r+radius;
                if(acc[y * width * radiiCount + x * radiiCount + r] / (4*round((currentRadius)*SQRT2)) * 100 > threshold) {(@\label{lst:hough_if}@)
                	// Add circle
                	...
                	// Clear circle from hough space
                	...
                }
            }
        }
    }
}
\end{lstlisting}

\begin{figure}[htb]
	\centering
	\begin{subfigure}{.5\textwidth}
		\centering
		\includegraphics[width=.8\linewidth]{Bilder/canny.png}
		\caption[Eingabebild der Circle Hough Transformation]{Eingabebild der \ac{CHT}}
	\end{subfigure}%
	\begin{subfigure}{.5\textwidth}
		\centering
		\includegraphics[width=.8\linewidth]{Bilder/hough.png}
		\caption[Ausgabebild der Circle Hough Transformation]{Ausgabebild der \ac{CHT}}
	\end{subfigure}
	\caption{Ein- und Ausgabebild der Circle Hough Transformation}
\end{figure}
