Als Interfaces für den Grayscaler dienen die Felder, welche Ein- und Ausgabedaten enthalten. Als einzige Anpassung am Code empfiehlt es sich, die Berechnung des Intensitätswertes auf eine ganzzahlbasierte Berechnung zu ändern.

\begin{equation}
output[index] = (30*r + 59*g + 11*b)/100
\end{equation}

Hierdurch verringert sich zwar die Präzision minimal, allerdings beträgt die Abweichung des Intensitätswertes höchstens 1 im Vergleich zur alten Berechnungsmethode. Dies ist in dem erzeugten Bild nicht erkennbar und spielt für die weitere Verarbeitung keine Rolle. Entscheidend ist, wie in \autoref{img:synthese_grayscaler400x400} zu sehen, dass der Ressourcenverbrauch 2.837 benötigten \acp{FF} und 3.017 \acp{LUT} auf 136 benötigte \acp{FF} und 194 \acp{LUT} sinkt. Zusätzlich wird die Anzahl der benötigten Takte um einen Faktor von circa 8,2 auf 640.801 Takte reduziert.

\begin{figure}[H]
	\centering
	\includegraphics[width=0.8\linewidth]{Bilder/synthese_grayscaler400x400.png}
	\caption[Ergebnisse der Synthese des Grayscalermoduls]{Ergebnisse der Synthese des Grayscalermodul bei einem 400x400 Pixel großen Bild. Erzeugt mit Vivado HLS 2017.4 \cite{vivado902ug}}
	\label{img:synthese_grayscaler400x400}
\end{figure}

Bei einem Pipelining der inneren Schleife kann ein Bild der Größe $hoehe \cdot breite$ in $hoehe \cdot breite$ Takten verarbeitet werden. Demnach kann ein Bild der Größe $400 \cdot 400$ in $160.000$ Takten verarbeitet werden. Um die äußere Schleife pipelinen zu können, muss die innere Schleife zunächst unrolled werden. Dies führt dazu, dass beim Pipelining $breite$-mal gleichzeitig auf das Feld $input$ zugegriffen wird. Das Feld wird allerdings über \ac{AXI4} bereitgestellt. Hierbei kann nur ein Zugriff pro Takt erfolgen, weshalb die äußere Schleife nicht gepipelined wird. Eine Lösungsmöglichkeit stellt das \emph{cachen} des Eingabefeld dar. Dabei wird das Feld in den \ac{BRAM} des \ac{FPGA} geladen. Jedoch rechtfertigt die zum Laden der Eingabedaten in den \ac{BRAM} benötigte Zeit und der höhere Ressourcenaufwand dieses Vorgehen nicht.
\\
\\
\autoref{img:synthese_grayscaler400x400} zeigt, dass die \ac{PL} ungefähr 160.005 Takte beim Pipelining der inneren Schleife benötigt. Bei einer Taktrate von 200 MHz, die als realisierbar betrachtet werden kann, wird nach der Synthese eine Laufzeit von 0,8ms und damit ein Speedup von 5 erzielt.
