Der Grayscaler wandelt ein RGB-Bild nach \autoref{sec:umwandlung-eines-rgb-bildes-in-graustufen} in ein Graustufenbild um. Dabei erhält er als Eingabe einen Zeiger auf 32 Bit breite Pixeldaten und die Höhe und Breite des Bildes. Anschließend wird jedes Pixel durchlaufen und in \autoref{lst:grayscaler} Z. \ref{lst:grayscaler_output} der neue Intensitätswert berechnet. Die einzelnen Werte für Rot, Grün und Blau berechnen sich in Z. \ref{lst:grayscaler_rgb_start} - \ref{lst:grayscaler_rgb_end}. Als Pixelformat wurde \emph{ARGB8888} gewählt, welches in \autoref{img:argb-format} näher dargestellt ist. Die 8 höchstwertigen Bit enthalten den Alphawert des jeweiligen Pixels, die nächsten 8 Bit den Rotwert, die folgenden 8 Bit den Grünwert und die 8 niedrigsten Bit den Blauwert.

\begin{figure}[H]
	\centering
	\includegraphics[width=0.8\linewidth]{Bilder/ARGB_format.png}
	\caption[Pixelformat ARGB]{Darstellung des Pixelformates ARGB8888}
	\label{img:argb-format}
\end{figure}

Die For-Schleifen, in welchen das gesamte Bild durchlaufen und verarbeitet wird, werden in den Z. \ref{lst:grayscaler_omp_start} - \ref{lst:grayscaler_omp_end} parallelisiert, falls \ac{OMP} verwendet wird.

\begin{lstlisting}[label=lst:grayscaler,caption=Auszug aus grayscaler.c]
uint8_t* grayscaler(uint32_t* input, unsigned int width, unsigned int height) {
uint8_t *output = malloc(width * height * sizeof(uint8_t));

#ifdef _OPENMP(@\label{lst:grayscaler_omp_start}@)
#pragma omp parallel for
#endif(@\label{lst:grayscaler_omp_end}@)
// Convert image to grayscale
for(int y = 0; y < height; y++) {(@\label{lst:grayscaler_outer_loop}@)
	for(int x = 0; x < width; x++) {(@\label{lst:grayscaler_inner_loop}@)
		uint8_t r, g, b;
		int index = y * width + x;
		
		r = input[index] >> 16 & 0xFF;(@\label{lst:grayscaler_rgb_start}@)
		g = input[index] >> 8 & 0xFF;
		b = input[index] & 0xFF;(@\label{lst:grayscaler_rgb_end}@)
		output[index] = 0.3*r + 0.59*g + 0.11*b;(@\label{lst:grayscaler_output}@)
	}
}

return output;
}
\end{lstlisting}

Als Ausgabe liefert der Grayscaler einen Zeiger auf die neu berechneten 8 Bit breiten Pixeldaten. Es muss von der Main Methode dafür gesorgt werden, dass die erzeugten Pixeldaten auch in ein geeignetes Bild geladen werden.

\begin{figure}[H]
	\centering
	\begin{subfigure}{.5\textwidth}
		\centering
		\includegraphics[width=.8\linewidth]{Bilder/euro.png}
		\caption[Eingabebild des Grayscalers]{Eingabebild des Grayscalers\\ $[$http://www.historia-hamburg.de/media/product/1ec/19-x-1-euro-satz-aus-19-euro-staaten-511.jpg$]$}
	\end{subfigure}%
	\begin{subfigure}{.5\textwidth}
		\centering
		\includegraphics[width=.8\linewidth]{Bilder/grayscale.png}
		\caption[Ausgabebild des Grayscalers]{Ausgabebild des Grayscalers\newline\newline\newline}
	\end{subfigure}
	\caption{Ein- und Ausgabebild des Grayscalers}
\end{figure}
