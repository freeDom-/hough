Die Canny Edge Detection erzeugt ein binäres Gradientenbild, welches für die \ac{CHT} benötigt wird. Dies wurde bereits in \autoref{sec:canny-edge-detection} erläutert. Sie bekommt als Argumente einen Zeiger auf die 8 Bit breiten Pixeldaten, die Höhe und Breite des Bildes, sowie zwei Schwellwerte übergeben.
\\
\\
Die beiden Schwellwerte werden bei der Hysterese genutzt, um auch schwache Kanten, welche über dem niedrigeren Schwellwert liegen, aber mit starken Kanten, welche über dem höheren Schwellwert liegen, verbunden sind, im Kantenbild zu behalten.
\\
\\
Als Ausgabe wird ein Zeiger auf die 8 Bit breiten Kantendaten geliefert. Es muss von der Main Methode dafür gesorgt werden, dass die erzeugten Pixeldaten auch in ein geeignetes Bild geladen werden.

\begin{figure}[H]
	\centering
	\begin{subfigure}{.5\textwidth}
		\centering
		\includegraphics[width=.8\linewidth]{Bilder/gauss.png}
		\caption{Eingabebild der Canny Edge Detection}
	\end{subfigure}%
	\begin{subfigure}{.5\textwidth}
		\centering
		\includegraphics[width=.8\linewidth]{Bilder/canny.png}
		\caption{Ausgabebild der Canny Edge Detection}
	\end{subfigure}
	\caption{Ein- und Ausgabebild der Canny Edge Detection}
\end{figure}
