Die Canny Edge Detection erzeugt ein binäres Gradientenbild, welches für die \ac{CHT} benötigt wird. Dem Modul werden als Argumente ein Zeiger auf die acht Bit breiten Pixeldaten, die Höhe und Breite des Bildes, sowie zwei Schwellwerte übergeben.
\\
\\
Für die Kantenerkennung werden mehrere Blöcke an verschachtelten For-Schleifen durchlaufen. Diese durchlaufen jedes Pixel, um den horizontalen Gradienten $g_x$ und den vertikalen Gradienten $g_y$ zu berechnen. Anschließend wird für jedes Pixel der Gradient berechnet und schließlich noch einmal jedes Pixel mit Hilfe von verschachtelten For-Schleifen für die \emph{non maximum suppression} und Hysterese durchlaufen.
\\
\\
Als Ausgabe wird ein Zeiger auf die acht Bit breiten Kantendaten geliefert. \autoref{img:canny_result} zeigt die Ausgabe des Canny Edge Detection Moduls.

\begin{figure}[H]
	\centering
	\begin{subfigure}{.5\textwidth}
		\centering
		\includegraphics[width=.8\linewidth]{Bilder/gauss.png}
		\caption{Eingabebild der Canny Edge Detection}
	\end{subfigure}%
	\begin{subfigure}{.5\textwidth}
		\centering
		\includegraphics[width=.8\linewidth]{Bilder/canny.png}
		\caption{Ausgabebild der Canny Edge Detection}
	\end{subfigure}
	\caption{Ein- und Ausgabebild der Canny Edge Detection}
	\label{img:canny_result}
\end{figure}
