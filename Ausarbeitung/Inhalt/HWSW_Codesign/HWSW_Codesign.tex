Bei dem Zynq UltraScale MPSoC besteht das Hardware/Software Codesign aus dem Zusammenspiel des Cortex-A53 Prozessors und des Zynq UltraScale XCZU9EG-2FFVB1156 \acp{FPGA}. Die Hauptanwendung läuft auf dem Prozessor, während einzelne Komponenten zur Beschleunigung auf den \ac{FPGA} ausgelagert werden. Das Hardware/Software Codesign wird mit dem Programm Vivado HLS, aus der Vivado Design Suite 2017.4 \cite{vivado902ug}, durchgeführt. Dazu wird im Folgenden zunächst die Softwareimplementierung und anschließend die Auslagerung mittels \ac{HLS} vorgestellt.
\\
\\
Jedes Hardwaremodul benötigt Schnittstellen für den Datenaustausch. Diese werden Interfaces genannt. Vivado HLS definiert standardmäßig alle Argumente einer Funktion als Interfaces, über die Daten empfangen oder gesendet werden können. Außerdem können im Programm Schnittstellen über Direktiven definiert werden. Für die Kommunikation zwischen \ac{CPU} und \ac{PL} wird oftmals das \ac{AXI4} Protokoll verwendet \cite[S. 167ff]{vivado902ug}. Bei diesem können mehrere Schnittstellen zu einem Bus zusammengefasst werden. Die \ac{HLS} kümmert sich dann um das Bereitstellen der benötigten Schnittstelle. Bei der Implementierung des \ac{AXI4} Moduls werden automatisch C-Treiber generiert. Diese stellen ein \ac{API} zur Ansteuerung des synthetisierten Moduls bereit.
