Mit dem Gauß-Filter wird das Bild, wie in \autoref{sec:gaus-filter} beschrieben, zur Kantenerkennung vorbereitet. Dazu bekommt die Funktion einen Zeiger auf die 8 Bit breiten Pixeldaten, die Höhe und Breite des Bildes, so wie die Kernelgröße als Argumente übergeben.
\\
\\
Mit Hilfe des Pascal'schen Dreiecks wird zunächst ein eindimensionaler Filterkernel generiert, welcher die Gauss'sche Funktion der Normalverteilung approximiert. Dieser wird zuerst in horizontaler und anschließend in vertikaler Richtung auf die Pixeldaten angewandt. Die Anwendung in vertikaler Richtung erfolgt dabei auf die zuvor horizontal gefilterten Daten. Dabei werden die Daten zuerst, wie in \autoref{sec:gaus-filter} beschrieben, gewichtet aufsummiert und anschließend normiert. Bei der Normierung wird durch die Summe aller Kernelelemente geteilt, da diese aber auf Grund der Approximation durch das Pascal'sche Dreieck eine Zweierpotenz ergibt, wird als Optimierung ein Rechtsshift um $2\cdot(Kernelgr"o"se-1)$ durchgeführt. Das Bild wird, wie in \autoref{sec:gaus-filter} beschrieben, an den Ecken um die benötigten Pixel erweitert.
\\
\\
Als Ausgabe wird ein Zeiger auf die gefilterten 8 Bit breiten Daten geliefert. Es muss von der Main Methode dafür gesorgt werden, dass die erzeugten Pixeldaten auch in ein geeignetes Bild geladen werden.

\begin{figure}[htb]
	\centering
	\begin{subfigure}{.5\textwidth}
		\centering
		\includegraphics[width=.8\linewidth]{Bilder/grayscale.png}
		\caption{Eingabebild des Gauß-Filters}
	\end{subfigure}%
	\begin{subfigure}{.5\textwidth}
		\centering
		\includegraphics[width=.8\linewidth]{Bilder/gauss.png}
		\caption{Ausgabebild des Gauß-Filters}
	\end{subfigure}
	\caption{Ein- und Ausgabebild des Gauß-Filters}
\end{figure}
