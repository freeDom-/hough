Die Auslagerung des \ac{CHT} Moduls erfolgt analog zu der des Grayscalermoduls, das in \autoref{sec:implementierung-grayscaler} beschrieben ist.
\\
\\
\autoref{img:implementierung_utilization_hough400x400} ist zu entnehmen, dass die Implementierung der \ac{CHT} die Werte für den Ressourcenverbrauch einhält. Die Timings können nach der Implementierung ebenfalls eingehalten werden, sodass eine Taktrate von 200 MHz realisiert werden kann.

\begin{figure}[H]
	\centering
	\includegraphics[width=\linewidth]{Bilder/implementierung_utilization_hough400x400.png}
	\caption[Ressourcenaufwand der Implementierung der Circle Hough Transformation]{Ressourcenaufwand der Implementierung der Circle Hough Transformation bei einem 400x400 Pixel großen Bild. Erzeugt mit Vivado 2017.4 \cite{vivado902ug}}
	\label{img:implementierung_utilization_hough400x400}
\end{figure}
%TODO: eigenes Diagramm einfügen?
