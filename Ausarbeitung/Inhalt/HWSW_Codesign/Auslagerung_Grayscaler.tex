Die Implementierung des Grayscalers beschränkt sich auf die Verwendung von vier \acp{IP}. Die Zynq UltraScale+ MPSoC \ac{IP} \cite{vivado201pg} enthält das \ac{PS} und die \ac{PL}. Die \ac{AXI4} Interconnect \ac{IP} \cite{vivado059pg} verbindet das Grayscalermodul, welches in der \ac{PL} realisiert wird, über einen Bus mit dem \ac{PS}. Als letzte Komponente wird ein Resetmodul \cite{vivado164pg} benötigt, welches das Reset-Signal für den Grayscaler erzeugt.
\\
\\
Das \ac{PS} kann über das von dem \ac{AXI4} Modul generierte \ac{API} angesteuert werden.

\begin{figure}[htb]
	\centering
	\includegraphics[width=0.8\linewidth]{Bilder/implementierung_blockschaltbild_grayscaler400x400.png}
	\caption[Blockschaltbild für die Implementierung des Grayscalermoduls]{Blockschaltbild für die Implementierung des Grayscalermodul bei einem 400x400 Pixel großen Bild. Erzeugt mit Vivado 2017.4 \cite{vivado902ug}}
	\label{img:implementierung_timing_grayscaler400x400}
\end{figure}

Aus \autoref{img:implementierung_timing_grayscaler400x400} und \autoref{img:implementierung_utilization_grayscaler400x400} geht hervor, dass die von der Synthese in \autoref{sec:synthese-grayscaler} berechneten Werte auch nach der Implementierung eingehalten werden können. Sowohl der Platzbedarf als auch das Timing liegen in einem zufriedenstellenden Bereich.

\begin{figure}[htb]
	\centering
	\includegraphics[width=\linewidth]{Bilder/implementierung_timing_grayscaler400x400.png}
	\caption[Timing der Implementierung des Grayscalermoduls]{Timing der Implementierung des Grayscalermodul bei einem 400x400 Pixel großen Bild. Erzeugt mit Vivado 2017.4 \cite{vivado902ug}}
	\label{img:implementierung_timing_grayscaler400x400}
\end{figure}

\begin{figure}[htb]
	\centering
	\includegraphics[width=\linewidth]{Bilder/implementierung_utilization_grayscaler400x400.png}
	\caption[Ressourcenaufwand der Implementierung des Grayscalermoduls]{Ressourcenaufwand der Implementierung des Grayscalermodul bei einem 400x400 Pixel großen Bild. Erzeugt mit Vivado 2017.4 \cite{vivado902ug}}
	\label{img:implementierung_utilization_grayscaler400x400}
\end{figure}
