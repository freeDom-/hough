Die Implementierung des Grayscalers beschränkt sich auf die Verwendung von vier \acp{IP}. Das Blockschaltbild in \autoref{img:implementierung_blockschaltbild_grayscaler400x400} zeigt die Verschaltung der \acp{IP}. Die Zynq UltraScale+ MPSoC \ac{IP} \cite{vivado201pg} enthält das \ac{PS} und die \ac{PL}. Die \ac{AXI4} Interconnect \ac{IP} \cite{vivado059pg} verbindet das Grayscalermodul, welches in der \ac{PL} realisiert wird, über einen Bus mit dem \ac{PS}. Als letzte Komponente wird ein Resetmodul \cite{vivado164pg} benötigt, welches das Reset-Signal für den Grayscaler erzeugt.

\begin{landscape}
	\begin{figure}[H]
		\centering
		\includegraphics[width=\linewidth]{Bilder/implementierung_blockschaltbild_grayscaler400x400.png}
		\caption[Blockschaltbild für die Implementierung des Grayscalermoduls]{Blockschaltbild für die Implementierung des Grayscalermodul bei einem 400x400 Pixel großen Bild. Erzeugt mit Vivado 2017.4\\\centering\cite{vivado902ug}}
		\label{img:implementierung_blockschaltbild_grayscaler400x400}
	\end{figure}
\end{landscape}

Aus \autoref{img:implementierung_utilization_grayscaler400x400} geht hervor, dass die von der Synthese in \autoref{sec:synthese-grayscaler} berechneten Werte bezüglich des Ressourcenverbrauches nach der Implementierung eingehalten werden können. Auch die erwarteten Timings werden eingehalten, sodass eine Taktrate von 200 MHz realisiert werden kann.

\begin{figure}[H]
	\centering
	\includegraphics[width=\linewidth]{Bilder/implementierung_utilization_grayscaler400x400.pdf}
	\caption[Ressourcenaufwand der Implementierung des Grayscalermoduls]{Ressourcenaufwand der Implementierung des Grayscalermodul bei einem 400x400 Pixel großen Bild.}
	\label{img:implementierung_utilization_grayscaler400x400}
\end{figure}
